% ATENÇÃO - veja com o seu orientador se você vai ter este capítulo e se este vai ter nome!
\chapter{Trabalhos Relacionados}
\label{cap:trabalhos:relacionados}

Essa seção faz uma breve revisão de trabalhos relacionados e semlhantes na área de Apicultura
de precisão.

Mundialmente, a maior parte da produção se dá em torno de abelhas com ferrão, refletida nos
autores citados através de sua unanimidade na escolha de espécies para pesquisa: _Apis mellifera_. No
Brasil, uma parte consideravelmente maior de produção existe graças a meliponicultura (cultura
de abelhas sem ferrão), cuja principal especime é a _Tetragonisca angustula_. Essa,
infelizmente, ainda não foi objeto de pesquisa de nenhuma referência apresentada.

Para o assunto, \cite{Terenzi2020} é de extrema importância. Com o objetivo principal revisar abordagens usadas na captura e
análise de áudios de melíferas, este cita alguns trabalhos importantes envolvendo a aquisição e armazenamento dos dados para
servidores usando tecnologias sem fio. Um dos citedos, \cite{Ferarri2008}, estudou o fenômeno
da **exameação**. Um total de 270h de áudio durante o acontecimento de 9 exameamentos foram
estudados. Fizeram o uso de microfones omnidirecionais e sensores de umidade e temperatura.
Após classificarem manualmente os trechos de sinais que representavam o evento de divisão, os
sons foram analisados em tempo e frequência, com o objetivo de correlacionar ambos os sinais
obtidos. A divisão altera os tons comuns reproduzidos dentro da colméia, de 150Hz para 500Hz
durante o swarming, além de quedas em temperatura e umidade. Outro trabalho de peso é
\cite{Papachristoforou2008}, que após inserir um dos predadores naturais de meliferas no
ambiente, foi capaz de detectar uma faixa de frequência diferente de diversos outros padrões
anteriores: 6kHz. Essa obra permitiu perceber a necessidade da captura de um espectro mais
amplo de sons.

**enxameação**: Processo de divisão da colméia, onde após o crescimento populacional da
colméia, indivíduos formam um novo ninho, rainha, e consequentemente, colméia completamente
independente.

\iffalse Referências: Trabalhos relacionando aspectos importantes dos áudios
[Terenzi2020](On the Importance of the sound emitted by Honney Bee hives)

[Ferarri2008](detec enxam),

[Papachristoforou2008](nova freq)
\fi

\cite{Eskov2011} por outro lado, fez uma análise estatística. Todos os sinais foram tratados
como ruído contendo porções de dados com conteúdo espectral específico e detectável. Usando um
banco de 36h/dia durante cerca de 6 meses (o equivalente a 2 estações), antes de qualquer análise, os dados foram
separados em fragmentos de 1s, **normalizados** e pré-processados usando __Procedure of Optimal
Linear Smoothing (POLS)__, através do método de Gauss. Por fim, as sequencias, com amplitudes
processadas e suas flutuações, formam rankeadas usando um sistema estatístico. A conclusão foi
a mesma: é possível a previsão do evento de swarming pela alteração nas flutuações.

\iffalse Referências: Nova abordagem na análise de áudios

[Eskov2011](noise with message)

\fi

\cite{Howard2013} trata de um problema já citado, a detecção de colônias sem rainhas, mas usa a
**transformada S** para extração de features, e comparou o resultado com espectrogramas e
transformada de Fourier. O resultado da análise considerando o uso de Self-Organizing Maps
conseguiu calssificar as amostras em todos os casos, salvo falhas devido ao uso de espécies
distintas.

\iffalse Referências: Detecção de colônias sem rainhas

[Howard2013](alg for quenless colonies)

\fi

Finalmente, \cite{Cejrowsi2018} fez a mesma detecção, dando atenção aos algoritmos de **Linear
Predictive Coding** para extração de features. Então as estimações do LPC, **t-distributed
stochastic neighbor embedding** para redução de dimensiionalidade, e SVM para classificação. As
mudanças foram percebidas com sucesso, mas o ponto interessante reside na necessidade de um
novo treinamento com a nova abelha rainha.

\iffalse Detecção de abelhas raínhas em colméia

[Cejrowsi2018](detection of a queen bee)

O silêncio não dói
o que dói é ficar quieto
Ter todos os amores pre explicar
Tantas histórias pra celebrar
E enterrar com o peso da falta de palavras
da desatenção
Da falta de compaixão
Todos os dias de minha vida
\fi

\cite{Nolasco2018} também é um trabalho de grande importância para a área, uma vez que este
trata de anotações relacionando aspectos importantes para se considerar durante o
desenvolvimento de sistemas de aprendizado de máquinas para reconhecimento de eventos sonoros
em colméias. Os resultados não foram eficientes para CNN quando comparadas à de SVM, mas isso
se deve, em parte, ao uso dos sistemas para leitura de sinais em ambientes diferentes dos
treinados.

\iffalse Referências: Investigação de métodos de machine learning para reconhecimento de sons

[Nolasco2018](Bee or not bee: INVESTIGATING MACHINE LEARNING APPROACHES FOR BEEHIVE SOUND RECOGNITION)

\fi
			
Por fim, \cite{Kilyukin2018} fez uma excelente comparação entre o trabalho com a engenharia de features
e tempo de treino. enquanto CNN não precisaram de nenhum tipo de processamento para extrair
features específicas, estas demoraram mais tempo de treino para funcionar de maneira eficaz.

\iffalse Referências: Comparação entre deep learning e standard machine learning

[Kilyukin2018](Toward Audio Beehive Monitoring: Deep Learning vs. Standard Machine Learning in Classifying Beehive Audio Samples)

\fi

\iffalse

		https://web.archive.org/web/20181213122708/http://www.cs.tut.fi:80/sgn/arg/dcase2017/challenge/task-acoustic-scene-classification-results
			Acoustuc scene classification competition: try to cassify test recording into one of 15
			predefined classes.

		Autores usaram "Mel Spectra", "Mel -frequency cepstral coefficients" e "Hilber Huang transform"

	Bee swarm Activity Acoustic Classification for an IOT-Based Farm service

		How to detect bee swarming from audio signal [9]

		Dados em [10] e [11]

		Possíveis features em [16] e [17]

		Open Source Beehives project [23]

		Features Extraction Applied to the Analysis of the Sounds Emitted by Honey Bees in a Beehive

	Identificação de abelha órfã

		Extr. de feat
			Short Time Fourier Transform [16]
			Mel-Frequency Cepstral coefficients [17]
			Wavelet dec[18]
			Hilbert-Huang transform[19]

		Atenção especial para WT

	Investigating machine learning approaches for beehive sound recognition

\fi

