%%%% CAPÍTULO 1 - INTRODUÇÃO
%%
%% Deve apresentar uma visão global da pesquisa, incluindo: breve histórico, importância e justificativa da escolha do tema,
%% delimitações do assunto, formulação de hipóteses e objetivos da pesquisa e estrutura do trabalho.

%% Título e rótulo de capítulo (rótulos não devem conter caracteres especiais, acentuados ou cedilha)
\chapter{Introdução}\label{cap:introducao}

\section{Considerações iniciais}\label{sec:consideracoesIniciais}

A apicultura no brasil teve inicio em [], e hoje o país representa cerca de []% da produção
mundial, com expectativas de crescimento para []% até []. Parte deste crescimento se deve à
informatização dos processos primários da indústria. Na apicultura, a informatização consiste
na utilização de técnicas e tecnologias eletrônicas, de forma não invasiva, para o
condicionamento da produção de mel, o que é chamado de Apicultura de precisão. Esse processo
pode resultar num aumento de []% na produção deste de mel, utilizando técnicas de
sensoriamento, bem como monitoramento, se corretamente aplicada.

Dentre as diversas formas de condicionar a produção de mel, algumas delas envolvem o controle
de variáveis de ambiente, como temperatura, umidade, posicionamento geográfico, flora
disponível, etc. []. Todos estes fatores interferem diretamente o comportamento das abelhas [],
acarretando numa relação direta entre ambiente e colméia. Estudos como [] e [] mostram que
abelhas são capazes de comunicar mensagens e organizar indivíduos no comportamento de enxame
usando o som das asas. Em outras palavras, padrões comportamentais são disseminados através de
som, e a análise desses podem permitir identificar alterações de estados aos quais um enxame é
exposto.

No entanto, no Brasil, a existência de estudos no processamento destes dados é pouco e esparso.
Logo, é interessante a elaboração de um banco de dados, para que se permitam maiores pesquisas
sobre o condicionamento desses insetos, o desenvolvimento de novas técnicas para o manejo
adequado, ou mesmo o aprimoramento de técnicas já existentes.

Desta forma, o presente trabalho tem por objetivo o estudo do comportamento de abelhas
nacionais (Tetragonisca angustula), através da coleta e amostragem de um banco de áudios; a
extração e seleção de features do banco; a avaliação e comparação dos processos de extração de
features de abelhas africanizadas (Apis mellifera); e a otimização dos mesmos métodos para uso
em abelhas nacionais;

Espera-se que as tarefas acima permitam a elaboração de um espectro rudimentar de comportamento
natural desse animal, com o objetivo final de criar uma CNN capaz de identificar estados não
ideais dentro de uma colméia.

\section{Objetivos}\label{sec:objetivos}

\subsection{Objetivo geral}\label{subsec:objetivoGeral}

Categorização de espectros de comportamento da T. angustula por CNN, otimizando as features
classificadoras de comportamento usados em A. mellifera para o uso em T. angustula

\subsection{Objetivos específicos}\label{subsec:objetivosEspecificos}

Captura remota do áudio

(opcional) Criação de um banco de áudio de T. angustula

(opcional) Criação de um banco de áudio de A. mellifera

Compreender métodos para escolha de features (MFCC)

Compreensão dos métodos de redução de dimensionalidade, e representação de dados

\section{Justificativa}\label{sec:justificativa}

A importância deste trabalho reside na necessidade de disseminar ideias que potencializem a
sustentabilidade da produção de mel. Mundialmente, existem grupos interessados na
especialização e fortalecimento das técnicas usadas, através de conferências e simpósios sobre
assuntos relacionados \cite{Waset2023}. Conforme mostra \cite{Ribeiro2022}, a apicultura de precisão é um termo crescente
de pesquisa. Os autores também apresentam uma visão geral das publicações relacionadas ao assunto, que
posiciona o Brasil em segundo lugar, até o ano de 2022. Comparativamente, na produção mundial de mel \cite{Atlas2023}, o Brasil é o 11º produtor (2020), demonstrando a potencialidade do brasil, que cresce em produção cerca de 2 a 10\% ao ano, fazendo uso de técnicas manuais de monitoramento.

Além disso, a aplicação de tecnologia na apicultura permite um desenvolvimento aliado ao
equilíbrio ecológico, levando em consideração que o condicionamento da produção de mel não
permite apenas a coleta abundante do recurso, mas também o fortalecimento das cadeias de
polinização: cerca de 73\% das plantas cultivadas no mundo são polinizadas por abelhas, dando
tração a mercados Canadenses e Americanos, como um exemplo[5]. As abelhas, portanto, além de
garantirem sua própria sobrevivência, são essenciais para o mundo como um todo, e a otimização
de seus processos significa trabalhar pelo equilíbrio natural da flora.

Outro ponto importante é que a apicultura e meliponiocultura no Brasil se dão, em sua maioria,
de forma manual. Produtores investem recursos na verificação in loco de apiários e
meliponários , com frequência semanal, o que eleva custos de produção, e interfere no
equilíbrio do enxame, expondo o produtor e a produção à riscos. De forma tangente, a otimização
da produção, levando em conta a suficiência de variáveis como localização, clima, contato com
outras espécies, etc., são reféns da experiência pessoal de cada produtor.

Portanto, sistemas que preveem a condição da colméia mediante alteração destas variáveis
diminuem o risco de erros. Em outras palavras, é possível otimizar o manejo usando técnicas
simples de sensoriamento e monitoramento.

Por fim, a cidade de Pato Branco possui seus próprios produtores, e projetos de extensão, como
o "Amigos das Abelhas". A proposta é estudar o manejo destes animais, fazendo no processo um
registro dos conhecimentos, e a disseminação da importância do mel para a região e a humanidade
como um todo.

```
Eu deveria falar que faz parte de um produto comercial em elaboração?
```

\section{Estrutura do trabalho}\label{sec:estruturaTrabalho}

Esta obra se organiza da seguinte maneira:

- A seção de "Introdução", dá um panorama geral do problema, elencando tópicos marginais
de importância, posição nacional e mundial no cenário da produção de mel e pesquisa,
além de uma breve arguição sobre a importância do tema.

- A seção de "Referencial teórico" compreende trabalhos tratando da fundamentação teórica do problema: a
capacidade das abelhas de se comunicarem, e sua responsividade a mundanças no ambiente,
simildaridade entre sons de diferentes espécies, uma coleção geral de referências tratando da
captura e tratamento de áudio de Abelhas Africanizadas (Apis mellifera), além da eficácia
destes métodos, e por fim, uma breve revisão de métodos de aprendizagem comumente utilizados
para tal.

- A seção de "Trabalhos relacionados" compreendem uma coleção de obras semelhantes à presente, abordando, quando não os mesmos tópicos, temas de
importância imediata ára a área.

- A seção de "Materiais e método" consiste numa visão geral da proposta, e quais serão os
passos tomados na tentativa de estabelecer todos os objetivos

