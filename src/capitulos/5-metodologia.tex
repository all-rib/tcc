%%%% CAPÍTULO 3 - MATERIAL E MÉTODOS

\chapter{Materiais e Métodos}\label{cap:materialemetodos}

Esta seção trata do relato de tarefas estabelecidas para cumprir o objetivo geral: Categorizar
espectros de comportamento da T. angustula usando CNN, através da otimização dos métodos
classificadoras utilizadas em _A. mellifera_ para o uso em T. angustula. Esse processo vai se
dar através das seguintes tarefas:

Busca ou criação de um banco de dados de A. mellifera, para que sejam feitos os devidos testes
dos estudos já estabelecidos;

Seleção dos algoritmos específicos de pré-processamento, extração de features, e inferências;

Elaboração, padronização e gegistro do ambiente de amostragem: colméias de T. angustula;

Elaboração e montagem do harware para captura de áudio, e outros dados importantes;

Captura, amostragem dos dados, e separação de amostras para treino, e teste;

Execução dos algoritmos selecionados nos dados amostrados;

Análise e comparação dos resultados

\section{Materiais}

Microfone inmp441 \cite{Inmp441};

ESP32 WROOM \cite{Esp32};

Módulo ESPsd \cite{Espsd};

Caixas de abelhas T. angustula, com colméia saudável;

Computador robusto para execução dos algoritmos; % (ver hw da sala v107)

Cartão de memória SD;

Conexão com a internet;

\section{Métodos}

A primeira e mais importante etapa reside na busca de um banco de dados já existentes
de A. mellifera. Existem projetos como \cite{Hiveeyes2023} e \cite{Osbeehive2023} que disponibilizaram dados em trabalhos
como \cite{Zgank 2019} e \cite{Cecchi2018}.

\iffalse Referências: Bases de dados

[Osbeehive2023](https://www.osbeehives.com/)

[Hiveeyes2023](https://hiveeyes.org/)

[Zgank2019](Bee Swarm Activity Acoustic Classification for an IoT-Based Farm Service)

[Cecchi2018](A Preliminary Study of Sounds Emitted by Honey Bees in a Beehive)

\fi

Após isso, serão implementados os algoritmos dos trabalhos anteriores, se atentando para a
fidelidade dos métodos usados, na tentativa de obter o mesmo resultado, e estudar a eficácia
dos respectivos. Essa etapa, no entanto, não exige hardware aplicado em campo, com a utilização
dos bancos já existentes. Os resultados serão comparados e por consequência, calibrados para
funcionar conforme os modelos originais.

A próxima etapa consiste na elaboração de um sistema simples de captura de áudios, em
://invensense.tdk.com/wp-content/uploads/2015/02/INMP441.pdf 
um primeiro momento, fazendo uso de um microfone como o INMP441, capaz de capturar todo
o espectro de audição humana, numa resolução de até 16kbps [achar datasheet do
sujeito]. Detalhes da escolha explicados na seção de discuções. Para a interface e captura do áudio do
microfone, será usado um ESP32 WROOM \cite{Esp32}, uma unidade bastante comum e
extremamente barata, com capacidade de conversão AD, e expansão do armazenamento físico
fazendo uso de um módulo expansor microSD. Serão usados cartões de memória para
o armazenamento persistente, e uma pequena amostra será também elaborada para testar o
procedimento de extração dos sons, que consiste:

Captura do áudio em WAV;

Conversão para formato compactado e sem perdas (FLAC) (opcional);

Gravação em armazenamento físico;

Visualização e validação dos dados em computador;

Alimentação do dado nos algoritmos testados;

Vale a pena ressaltar que as mesmas obras realizadas por \cite{Sendeske2022} realizam o
primeiro e o terceiro sub-passo citado, e um projeto anterior de minha elaboração
realiza segundo. Além disso, existe uma otimização a ser feita durante o processo de
captura, ou de conversão: o espectro de frequência das abelhas, como citado, não é o
mesmo da fala humana, mas chega próximo. Mas para aproveitar melhor o espaço de
armazenamento, já não abundante nos cartões de memória, será feita captura de apenas um
canal (o microfone captura em estéreo), e numa faixa de frequência menor que a nativa
(para exclusão de faixas "inúteis". como acima dos 17kHz). Isso permite uma redução
significativa no tamanho final do arquivo sem qualquer compressão, ou perda
considerável. O uso do algoritmo FLAC para compactação vem do mesmo projeto pessoal,
\cite{Ribeiro2023}, que demonstram estudos apontando que os métodos de compressão
usados em algoritmos com perda (MP3 por exemplo), podem atenuar, e até remover.
informações importantes. Além disso, aponta FLAC como a melhor taxa de compressão entre
os algoritmos sem perda.

Com os dados em mãos, o próximo passo é tratar e pré-processar os dados, aplicando
algoritmos de transformação, como os já citados, afim de destacar informações
importantes. Essa etapa envolverá, também, a separação das amostras em teste e treino,
mantendo assim a confiabilidade do sistema.

Por fim, resta a execução dos algoritmos à exaustão, fazendo as devidas comparações. O
processo é iterativo, então, além da pesquisa de possíveis features, a otimização do
método exige o reprocessamento dos dados, quando não a recaptura dos mesmos em
condições diferentes.  Caso não surtam os resultados esperados, será feita ainda um
registro do sensoriamento de umidade e temperatura, integrado os 2 simultaneamente no
dht22 \cite{Dht22}.

\iffalse Referências:datasheets

[Esp32](https://www.espressif.com/sites/default/files/documentation/esp32-wroom-32_datasheet_en.pdf)

[Inmp441](https://invensense.tdk.com/wp-content/uploads/2015/02/INMP441.pdf)

[Dht22](https://www.sparkfun.com/datasheets/Sensors/Temperature/DHT22.pdf)

\fi

