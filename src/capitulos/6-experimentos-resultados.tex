%%%% CAPÍTULO 6 - DISCUSSÃO

\chapter{Discussão}\label{cap:discussao}

Primeiramente, existe a possibilidade de que os projetos citados como referência para
banco de áudio desses animais não estejam mais funcionando, ou mesmo os dados estejam
indisponíveis. Neste caso, existe um corpo de apoio, o projeto Amigos das Abelhas, na
universidade, capazes de auxiliar com a elaboração do mesmo.

Segundo, o processo de captura em WAV e gravação em armazenamento físico já foram
elaborados e testados por \cite{Ribeiro2022}, se atentando para o uso do mesmo
hardware. A ideia é evitar o retrabalho, além de otimizar o anterior.

Terceiro: A escolha do microfone se dá em função do mesmo motivo: o uso deste em projetos
anteriores realizados na matéria de Oficina de Integração, no final ano de 2022. além
disso, quando comparado a outros do mercado, tem excelente custo benefício.

Por fim, existe uma otimização impotante a ser feita caso a conversão para FLAC seja
necessária: na falta de armazenamento, FLAC servirá como compressor, evitando perda de
dados. No entanto, para uma maior redução, faixas de frequência acima de 17kHz podem
ser excluidas, se for comprovado que esta faixa não contém dados importantes, conforme
algumas referências já citadas. Também vale a pena citar que usei o mesmo procedimento
em \cite{Ribeiro2023}, para a transmissão de áudios por LoRa, um protocolo extremamente
econômico em taxa de tranferência. No artigo são discutidos o porquê do uso do FLAC, em
detrimento de outros formatos como MP3, ou AAC.

\iffalse

[RibeiroAcco2022](Sistema de Monitoramento de Api{\' a}rios Utilizando Capta{\c c}{\~ a}o Sonora e LoRa)

[Ribeiro2023](Compacta{\c c}{\~ a}o de {\' A}udio e Transmiss{\~ a}o por LoRa)

\fi
