%%%% CAPÍTULO 2 - REVISÃO DA LITERATURA (OU REVISÃO BIBLIOGRÁFICA, ESTADO DA ARTE, ESTADO DO CONHECIMENTO)
%%
%% O autor deve registrar seu conhecimento sobre a literatura básica do assunto, discutindo e comentando a informação já publicada.
%% A revisão deve ser apresentada, preferencialmente, em ordem cronológica e por blocos de assunto, procurando mostrar a evolução do tema.
%% Título e rótulo de capítulo (rótulos não devem conter caracteres especiais, acentuados ou cedilha)
\chapter{Referencial te\'orico}\label{cap:referencialTeorico}

As abelhas são animais com uma forte relação de enxame, sendo capazes de identificar indivíduos, definir
funções, tomar decisões coletivas, e até de lembrar []. Esses indivíduos, um dentre os mais inteligentes
dentre espécies semelhantes, são seres extremamente sociais, de forma que nenhuma colméia prevalece sem
constante distribuição de mensagens e execução de tarefas específicas. Desde [] sabe-se que as abelhas
são capazes de comunicar mensagens atrávés da vibração do abdomen e das asas. E todas as espécies, sem
exceção, podem comunicar a alteração de variáveis do ambiente, permitindo uma adaptação coletiva às novas
condições[]

Consequência disso é que é possível gerar sons artificiais semelhantes aos que estes animais geram, para
induzir comportamentos no enxame, permitindo que certos estados sejam provocados, conforme mostra \cite{Michelsen1986}. O
mesmo autor ainda compara 3 espécies diferentes de abelhas, encontrando semelhança nos padrões sonoros, o
que permitiu a previsão de comportamento num futuro breve, baseando-se apenas na identificação desses
padrões.

Por consequência, vários estudos relacionados à aquisição de dados, relacionando variáveis do ambiente
com padrões comportamentais, foram elaborados.

\cite{Wenner1964} explica que as abelhas não são capazes de enxergar no escuro, e portanto, "dança" não
seria a única ferramenta responsável pela transmissão de informação dentro de uma colméia. O autor
identificou, com sucesso, a transmissão de informações relacionadas à distância de uma fonte de alimento.
O autor ainda faz distinção de dois tipos de sons: "Characteristic hum of a beehive" e "disturb",
respectivamente, sons produzidos como consequência da ventilação da colméia, provocada por abelhas
operárias, e sons para alerta de intrusão, produzido por abelhas sentinelas.

Também é sabido que existe uma relação direta entre os sons característicos da colméia para ventilação:
conforme as estações do ano passam, e as condições climáticas mudam, as necessidades de ventilação se
adaptam para tal. Isso é demonstrado por \cite{Dietlein1985}, que capturou várias amostras durante um
ano, observando maior atividade durante o verão, e atividade nula durante o inverno. Também notoou-se que
os sinais capturados ficavam abaixo de 600Hz.

Mas como a colméia interage com os sons? \cite{Frings1957} apresenta alguns exemplos. Estas criaturas
respondem em função da duração, faixa de frequência, e repetição de funções específicas de áudios, como
se fossem "frases sonoras". Na tentativa de induzir comportamentos, o autor cita que Hansson (d;.
Hansson, Opuscula Entomol. Suppl. 6 (1945): Nord. Bitidskr. 3, 68 (1951)) foi capaz de impor um estado de
estaticidade nas abelhas, de maneira que o apicultor foi capaz de extrair o mel sem a necessidade usual
de **fumigação**. Com a mesma ideia, \cite{Michelsen1986} foi capaz de especificar um alcance maior de de
frequência no qual é obtido uma reação: de 100Hz à 3kHz.

Portanto, é estabelecido um consenso geral de que: as abelhas agem diferente em condições de ambiente
diferentes, e estas condições podem ser mapeadas utilizando os áudios, que são uma resposta direta da
percepção destes animais, e os mesmos os usam para comunicar estas mudanças entre sí.

**fumigação** - processo de exposição das abelhas a fumaça. Isto as faz entrar em estado de alerta contra
incêncio, as obrigando a se encherem de mel para possível abandono da colônia por incêndio.

Em se tratando do uso de computação para medir estes efeitos, o reconhecimento de padrões é uma atividade
bastante comum. As aplicações mais visíveis da área tratam do reconhecimento de um padrão com o objetivo
de detectar perturbações e outros sinais anômalos. A aplicação também pode envolver a adequação de
amostras à padrões já reconhecidos, ou em outras palavras, retificação e reconhecimento de sinais
ruidosos. \cite{Gales2007} demonstra exemplos desse, utilizando "Modelo oculto de Markov", ou "Hidden
Markov Model" (HMM) para reconhecer fala humana. HMM é um modelo estatístico, que mapeameia sistemas com
estados não observáveis através de parâmetros observáveis.

O autor faz uma revisão de como a arquitetura do sistema é composta, e apresenta melhorias para o
processo, dando atenção para a extração de "features". Features são aspectos de um sinal capazes de
distinguir dois sinais diferentes, permitindo assim uma análise mais objetiva das entradas de um dado
sistema. Dentre os mais usados, "Mel-Frequency Cepstrum Coefficients" (MFCC) é o mais simples.  a Eslaca
de Mel é uma escala que aproxima a distinção humana de tons sonoros através de uma equação. Algumas
constantes estão associadas a esta função, e quando definidas são chamadas MFCC, e a escala passa a ser
adaptável e reprodutiva. em \cite{Zheng2001}, o método é usado para identificar sílabas da língua
chinesa, otimizando os filtros e a seleção dos coeficientes para tal, obtendo a taxa de erros por sílaba
em 10%.

