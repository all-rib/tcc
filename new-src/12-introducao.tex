\chapter{Introdução}\label{cap:12}

Este capítulo descreve a importância da apicultura, conceitos místeres
para a proposta, a proposta, e finalmente, seu objetivo.

\section{Considerações iniciais}\label{cap:120}

A apicultura é parte essencial da agroindustria brasileira.
Responsáveis por 73\% da flora nacional, as abelhas são
imprescindíveis para um equilíbrio natural, produção de alimentos, e
preservação do meio ambiente[1]. Em 2022, a Associação Brasileira de
Estudo das Abelhas (A.B.E.L.H.A) publicou o Atlas da Apicultura no
Brasil[2]. Esta ferramenta reúne dados de instituições como a
Organização das Nações Unidas da Alimentação e Agricultura (FAO) e o
Instituto Brasileiro de Geografia e Estatística (IBGE). FAO aponta o
Brasil como 22º produtor mundial, tendo a produção aumentado em cerca
de 12\% entre 2021 e 2022.[3] Um crescimento significativo, mesmo
durante a pandemia. 

Paralelo a isso, existe um processo corrente que consiste na
informatização da produção de mel, utilizando tecnologias para
otimizar a extração do mel, em quantidade e qualidade, simultaneamente
reduzindo o impacto causado pela atividade, que atualmente é realizada
manualmente. A chamada Apicultura de Precisão utiliza sensores,
modelos de comportamento e microgerenciamento[4], uma ideia já
aplicada na agricultura de precisão. Porém é um tema novo no Brasil,
quando comparado a outros países pelo mundo[4].

Em pesquisas no corpo nacional de publicações[4], os maiores esforços
ainda desconsidera em grande parte a utilização das técnicas de
controle e monitoramento de sistemas para aplicação direta na produção
de mel, seja pela falta de dados no que tange às espécies nacionais de
abelhas, seja na falta de esforços especializados na área.

O estudo da eficácia da manipulação das colmeias, bem como otimização
do processo produtivo de mel, dada a potencialidade produtiva do
Brasil, também tem ganhado tração, que se evidencia no trabalho de
organizações como a Associação Brasileira de Estudo das Abelhas
(A.B.E.L.H.A). 

	A ideia não é nova. A apicultura é tópico de interesse para
pesquisas desde meados dos anos 90, e mundialmente tem ganhado muita
atenção graças à agricultura de precisão, originando o termo
apicultura de precisão (AP).[13]. Contudo, As pesquisas nacionais não
demostram a problemática oculta da metodologia de aplicação da AP.  Os
projetos descrevem modelos de sensoriamento e monitoramento, detalhar
problemas de posicionamento dos equipamentos, problemas previstos no
uso de circuitos em contato com estes animais, ou mesmo a razão de se
utilizar determinada metodologia em detrimento de outra.

Portanto, o presente trabalho se propõe elaborar um sistema de
onitoramento para colméias, expondo problemas.

% TODO: REMOVER - (e desenvolvendo
% técnicas adequadas para interfacear o uso da tecnologia para a
% extração de um recurso tão rico quanto o mel)

O sistema, simples, minimamente invasivo e resiliente, precisa ter
facilidade de instalação, manutenção, e telemetria, para fazer berço
aos futuros projetos relacionados, expondo, principalmente, os
percalços da amálgama entre matéria eletrônica e orgânica. Também
precisa ser testado para diferentes espécies, configurando assim uma
base para projetos semelhantes num futuro próximo.

\section{Objetivos}\label{sec:1200}

\subsection{Objetivo geral}\label{subsec:12000}

% TODO: Discutir novo titulo:
Estudos, aplicação e adaptação de métodos de monitoramento de apiários
em abelhas T. angustula


% Depois de referencial teorico, falar sobre "Trabalhos relacionados"

% Rever sobre materialw e metodos. A ideia é que falta um caitulo antes dizendo o que você fez, e depois em métodos explicar qual o metodo que tu usou pra falar que funciona. Talvez arquitetura? Metodologia? 

\subsection{Objetivos específicos}\label{subsec:objetivosEspecificos}

  %0 Mudar objetivos especificos - o que foi entregue?
  %1 
  %2 Cada objetivo especifico precisa de uma entrega

Implementação de uma arquitetura capaz de sensoriar variáveis
pertinentes à abelhas;

Teste da plataforma em contato com uma colméia;

Proposição e implementação de soluções para os problemas encontrados
durante o processo;

% TODO: Reler justificativa

\subsection{Justificativa}\label{sec:1201}

As recentes pesquisas, tanto nacionais, quanto mundiais, mostram-se
interessadas em grande parte na otimização da produção de alimentos.
As abelhas, como integrante essencial do processo produtivo, bem como
produtoras diretas de alimento, tem recebido atenção com as pesquisas
e o surgimento da área da AP. Nacionalmente, no entanto, estas ideias
ainda estão germinando.

Esse trabalho pretendee implementar e adaptar as metodologias já
exploradas, e aplicá-las numa espécie nacional, em especial a T.
angustula.

Especificamente nesta obra, a implementação de um sistema de
monitoramento em uma colméia de abelhas nativas possui interesse da
produção local de mel, visivelmente crescente em projetos como Amigos
das Abelhas sem Ferrão (Amigos das ASF), e auxílio no estudo de
prototipagem para a empresa Geoponica Tecnologia.

O projeto Amigos das ASF tem como objetivo propagar conhecimentos
sobre a importância e benefícios de preservação das espécies nativas
de abelhas sem ferrão, preservação da natureza, e cuidados com o meio
ambiente.

Por outro lado, a empresa Geoponica possui esforços na mesma linha,
desenvolvendo pesquisas para prototipar placas de monitoramento de
colméias a fim de detectar eventos anormais. O reconhecimento desses
eventos pode proporcionar ao apicultor a base para uma tomada rápida
de ações, otimizando a produção, preservando a saúde das abelhas, e
gerando mais alimento.

A documentação da aplicação final será disponibilizada ao projeto
amigos das ASF, e à empresa Geoponica, para que sejam feitas
plataformas de monitoramento à partir dos conhecimentos obtidos.

\section{Estrutura do Trabalho}\label{sec:1202}

Este trabalho organiza-se em 5 capítulos

	- Introdução, onde explica-se as motivações, objetivos e
justificativas para o desenvolvimento deste trabalho;

	- Referencial teórico, onde explora-se os conceitos básicos
utilizados no projeto, detalhando a importancia de cada conceito;

	- Metodologia, onde descreve-se o sistema proposto, as tarefas realizadas, e os
problemas observados.

	- Materiais e métodos, que demonstra quais ideias e procedimentos
serão implementados para correção de problemas, bem como os parametros
para aferir a efetividade das soluções;

	- Conclusão, que discute os resultados obtidos, e próximos passos
para trabalhos futuros;

