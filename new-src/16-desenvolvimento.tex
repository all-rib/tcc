Considerando os objetivos do projeto, definiu-se as seguintes etapas:

Busca de metodologias de sensoriamento de abelhas A. mellifera,
considerando-se os dispositivos mais comumente utilizados (sensor de
temperatura, umidade, pressão, microfone, e luminosidade). A busca
será feita através de ScienceDirect e Scholar Google. Esta etapa
nomea-se "Busca".

Aplicação dos métodos de monitoramento, e extração dos dados, em
colméias de T. angustula, observando principalmente a necessidade de
adaptação dos outros métodos, e exposição dos problemas
destes métodos. Essa etapa nomeia-se "Implemetação de Arquitetura".

Análise do processo de aplicação e problemas observados, para
elaboração e implementação de soluções, e discussão das mesmas. Essa
etama será "Resultados e discussões"

\subsection{Busca}

Foram buscado os termos colméia, monitoramento, abelha, modelo,
sensoriamento, precisão, apicultura, observando-se suas respectivas
traduções para o inglês (honeycomb, monitoring, bee;honeybee, model,
sensoring, precision, apiculture,). Adiante listam-se as considerações
mais importantes encontradas.

[4] faz uma pesquisa extensiva nos mesmos tópicos, provendo uma
revisão de literatura abrangente sobre o assunto. Suas pesquisas citam
a importância da umidade para a saúde geral da colméia, e manutenção
das crias, regulação térmica, prevenção da proliferação de fungos e
bactérias, e preservação das propriedades do mel para consumo das
crias. Além disso, a temperatura possui funções semelhantes, de
maneira que a maioria das operarias realizam atividades para
manutenção constante dessas variáveis.

Segundo a mesma autora, o som também possui uma enorme importância,
pois é um dos canais de comunicação para ações do enxame como um todo,
responsável pela transmissão de eventos (alertas sobre invasores,
fontes de alimento, intempéries climáticas) e comportamento sazonal
(como aumento de atividade de evaporação de umidade, ou enxameação).
Outras variáveis como peso e composição atmosférica também interferem
nas características de ruído da colméia.

Algumas metodologias de sensoriamento envolvem Machine Learning,
Análise de dados e visão computacional, a exemplo, [trindade,2018]
utiliza um sensor de temperatura e umidade, alem de celulas para medir
peso. Para transmissao dos dados, foi utilizado um dispositivo Xbee,
mas nao descreve detalhes sobre qualidade de sinal, ou mesmo sobre o
tipo de servidor utilizado, apontando apenas que os dados foram
processados utilizando-se microsoft excel.

Estendendo a analise da obra de maneira hipotetica, pode se estipular
que parte da decisao de utilizar um dispositivo Xbee foi devido a sua
facilidade de conexao com a rede, por ser facilmente interfaceavel com
outros sistemas operacionais. Torna-se então evidente a necessidade de
um sinal de telefonia movel, ou sinal de internet para a transmissao
dos dados para nuvem.

Alem disso, trabalhos como [ferrari,2008] possuem uma taxa de captura
muito maior de dados, estendendo não apenas variáveis como temperatura
e umidade, mas também microfone, e imagem, o que cria um enorme fluxo
de dados para serem processados, inviabilizando o processamento local.
Para endereçar esse problema, armmazenimento externo será utilizado.

% TODO: Sera que preciso encontrar algo comprovando que a metodologia
% de sensoriamento exige ter uma amostra como sendo "normal", e outra
% de comparação?

Portanto, percebe-se que um caminho para a criação de um perfil pode
envolver temperatura, som, umidade, pressão, características que podem
refletir eventos como a presença ou a falta de alimento [fonte], a
presença de predadores, níveis de saúde, e eventos específicos como a
enxameação. 

\subsection{Elaboração da arquitetura}

O sistema proposto, idealmente, precisa ser capaz de amostrar ao menos
um ponto dentro, e outro fora da colmeia, permitindo assim observar o
contraste entre os meios, uma analogia da experimentação científica,
configurando uma "amostra de controle", e um grupo experimental,
tonando possível a análise comparativa.

Para temperatura, foram utilizado os sensores SHT30, DS18B20, BME280,
e o DHT11, dois a dois, sendo um externo e outro interno.

Estes sensores foram escolhidos conforme disponibilidade e precisão. O
DHT11 é o mais simples e comumente utilizado em projetos hobistas, que
apesar de não confiável para aplicações que exigem precisão e
constância, são suficientes para medidas pontuais de temperatura e
umidade. O SHT30, o DS18B20 e o BME280 foram utilizados após problemas
de amostragem com o DHT11, devido à frequência e interferências
provocadas pela interação com matéria orgância. Além disso, o suporte
(como datasheet, source-code do fabricante e notas de aplicação), dá
suporte a verificação de integridade CRC. O DS18B20 é especialmente
interessante, pela possibilidade de contato direto com o mel por não
posuir quaisquer ponto para entradas de material organico.

Para processamento e armazenamento das informacoes, foi utilizado uma
placa [LORA32](https://github.com/LilyGO/ESP32-Paxcounter) da LILYGO,
devida ao seu hardware integrado para uso de cartoes SD, o que permite
a gravacao dos dados durante um grande periodo, dando suporte à até
16GB de armazenamento externo.

O dispositivo possui recomendações de alimentação de 5V pela porta USB
[17]. As [baterias utilizadas](fonte) são do tipo LiPo, de 11.1V e
5000mAh, comuns em projetos de robótica competitiva. Possui 3.7V por
célula, de maneira que duas das células foram utilizadas, totaluzando
7.4V. A sobretensão não apresentou quaisquer problemas no controlador,
seja em testes de processamento ou de consulta de periféricos, com o
único sintoma observado sendo o aquecimento do regulador de tensão
presente na placa.

[Imagem lora32]

Para integração dos sensores, cabeamento de rede do tipo cat6 foi
utilizado, por sua disponibilidade, excelente condutividade, e
capacidade de redução de efeitos de interferencia, um problema comum
na utilizacao de conectores para a interface I2C[15].

Para adaptação dos cabos CAT6, foi necessario uma pequena topologia
para interligar os sensores à placa. Outro ponto a ser citado é a
dificuldade em alimentar os sensores utilizando apenas as saídas da
placa. Foi necessário a utlização de um regulador de tensão,
controlado digitalmente pelo ESP, para alimentação independente dos
sensores.

[Imagem topologia]

Num primeiro experimento, um total de 9 sensores foram conectados,
sendo um microfone, e, dois de cada um dos sensores a seguir: DHT11
(mais tarde descartado, por sua baixa confiabilidade e
incompatibilidade com I2C), SHT30, BME280 e DS18B20.

Os sensores foram conectados em paralelo, permitindo que sejam
consultados sob demanda. Graças à arquitetura I2C, mais de um sensor
do mesmo tipo pode ser utilizado na mesma linha: Os dispositivos são
endereçados de fábrica, e um ou mais bits podem ser alterados
manualmente. O endereçamento é determinado por 7 bits (0b111011x),
sendo os 6 MSB (most-significant-bits) constantes, e o menos
significativo controlado pelo nivel lógico da porta SDO. A conexão
entre SDO e GND força o endereço 0b1110110 (0x76), e a conexão com
VDDIO resulta em 0b1110111 (0x77). O pino não pode flutuar.

Se necessário, mais de dois dispositivos do mesmo tipo poderiam ser
conectados, aproveitando um detalhe da arquitetura SPI, que utiliza um
terceiro sinal (denominado CS, ou chip-select) para controlar a
seleção de um dispositivo. Neste caso, o sinal SDO tem a mesma função
do CS, mantendo todos os dispositivos com o mesmo endereço (0x76).
Quando for necessário consulta, a linha SDO deve sofrer pull-up,
alterando o endereço para 0x77, destacando-o como tendo um endereço
exclusivo, e portanto responderá individualmente. Ao final da
operação, a linha deve novamente sofrer pull-down, liberando o canal.

Os sensores, interligados entre si, em um primeiro teste foram
posicionados fora do mel, no sobreninho mais superior, o mais distante
do contato com os favos, prevendo possiveis problemas de funcionamento
dos sensores devido exposição ao mel e a cera.

% TODO: Falar sobre o teste de mergulhar o sensor no mel.

Foi necessário selar quaisquer pontos de conexão e passagem de fios
dentro e no limiar da colméia.  isso foi necessário devido à atividade
das abelhas de selar quaisquer buracos, e esterelizar materiais
internos, cobrindo-os com uma liga de cera e própolis. Esse processo
acontece devido ao ataque de outros insetos, em especial formigas e
fungos, que se aproveitam dos pontos de abertura que os cabos
provocam.  Além disso, corpos estranhos também podem estar infectados,
ou possuir material noscivo às abelhas. Essa prática não danifica os
sensores, mas impediram a comunicação perfeita do sensor, além de
interferir de maneira dinâmica na captura destes dados, alterando
condutividade térmica, leitura absoluta dos dados, e alterando o meio
de propagação de ondas sonoras. Poucos dados foram capturados, tendo
em vista as interferências observadas.

Para o ponto de passagem dos cabos, no entanto, a cera é capaz de
selar a entrada dos fios, por se moldar naturalmente à passagem de
cabos, criando um canal justo, e facilmente renovável, além de ser
familiar para os indivíduos da colônia, e portanto benéfica.

A captura dos dados foram feitas através de consulta sob demanda de
cada sensor, utilizando-se em sua grande maioria o código fonte
fornecido pelo fabriante, e gravação dos dados ocorreu em
armazenamento utilizando uma função atômica de escrita, em um arquivo
de texto, conforme o padrão a seguir

[#n][Timestamp][Nome]([medida][unidade])\\+

Esse padrão foi utilizado por fornecer contexto de origem do dado,
instante no qual foi capturado, sequencia entre várias amostrasa, e
respectiva medida de cada sensor, consideradas suficientes para a
naturza desta aplicação.

Finalmente, o microcontrolador, bem como uma bateria de alimentação,
foram isolados entre sí, e inseridos em frascos de conserva, selando
pontos de passagem de cabos com resina do tipo epoxi, evitando
exposição à umidade. Os frascos foram dispostos de ponta cabeça,
evitando que água acumulasse proximos aos conectores. Um conector foi
deixado do lado de fora, para facilitar a troca de bateria, caso
necessário.

Os primeiros testes foram realizados fora da colméia, apresentando
problemas apenas no DHT11, que posteriormente foi deixado de lado. Os
sensores foram testados mergulhando os mesmos por inteiro dentro do
mel. Nenhum problema foi observado.

[Graficos dentro do mel]

Nos testes executados, o sistema rodou sem problema por três dias
ininterruptos. Posteriormente, em contato com as colméias, os seguintes problemas
foram observados:

Os sensores sofriam de constantes falhas nos sinais e conexões. A
suposição é de que o ambiente interno da colméia possui uma interação
distinta quando comparada ao sensor em contato com o mel. Além disso,
a utilização de resinas ou outros produtos para impermeabilizar estes
sensores interferem na sua precisão, e na saúde dos insetos, pois em
sua grande parte, eles possuem solventes e outros materiais tóxicos.

Para isso, a API padrão dos dispositivos ESP32 já possuem chamadas
para geração de logs, capaz de apresentar mensagens conforme a demanda
do usuário. Conforme explica [18], o sistema possui diversos níveis de
log, onde, ao configurar os parâmetros Log Level e Maximum Log Level,
podemos determinar quais mensagens aparecerão, variando, em ordem
decrescente de verbosidade, Verbose, Debug e Info até Warning, Error e
None. Caso alguma mensagem seja configurada com uma verbosidade maior
que Maximum Log Level, a mesma não constará no binário final.

A quantidade de eventos geradas era demasiada, e a manutenção das
mensagens (seja em remover, adicionar, ou alterar) tornou
impraticável o uso no mesmo modelo. Essas rotinas, aqui denominado
diagnóstico preventivo (em contraponto a um diagnóstico remediativo,
que alertaria após algum erro), apenas funciona com a capacidade de registro
dos mesmos. Na maioria das vezes, os erros observados impediam também
a devida escrita em armazenamento externo. Com o disparo das rotinas
de watchdog, a informação então era perdida.

Os eventos gerados também geravam concorrência no processo de escrita
no cartão. Além do sistema prosseguir tentando capturar os dados em
sensores com erros, mesmo a falha em escrever os dados provocavam mais
eventos, e por consequência, as rotinas paravam por completo.

Os erros no sistema, que por vezes disparavam watchdogs, drenavam a
bateria completamente. Isso ocorre em decorrência do alto
processamento inicial para inicializar o dispositivo e entrar em
regime de funcionamento. Os sensores, mesmo com suas funcionalidades
de economia de energia, quando precisavam ser inicializados
a cada nova

O diagnostico do sistema em campo é difícil, pela impossibilidade de
debug em tempo real, uso de osciloscópio, ou observação dos eventos em
tempo real.;

De maneira mais clara e detalhada, as interferências nos sensores
provocaram erros do tipo hard-fault, drenarando a bateria em minutos
caso o sistema não reiniciasse (por padrão o hard-fault possui um loop
infinito). Quando implementado um watchdog, era possível capturar o
erro mesmo em estado de hard-fault, quando não envolvesse estouro de
pilha, ou tentativa de acessar regiões de memórias não mais alocadas
(como consultar dispositivos que já falharam). Mas eventualmente o sistema
reiniciaria novamente até drenar a bateria por completo em algumas
horas. A intervenção do usuário, então, sempre seria necessaria.























% TODO: Deveria usar a ideia do Denardim, e criar uma região reservada
% para registro de logs?

% TODO: Ver como funciona o debug de um hard-fault dentro de um RTOS

Para efeitos práticos, um wrapper foi criado para automatizar a
geração das mensagens de log, dando origem à uma biblioteca denominada
HERMES. Esse wrapper foi escrito de maneira que quaisquer ações
direcionadas a um sensor gerasse eventos, e os registrasse em
armazenamento de massa.

A biblioteca possui o conceito de "object", ou "objeto", onde um objeto é
capaz de realizar determinadas tarefas, classificadas em inicializar,
configurar, consultar, e desinicializar. Essas tarefas são abstraídas
no conceito de "taskchain", ou "corrente de tarefas", onde um ou mais
métodos, ou funções, podem ser cadastradas na chain para realizar
chamadas subsequentes na ordem em que foram cadastradas. E para cada
nova chamada, um evento é gerado e armazenado.

Além disso, a biblioteca foi implementada com uma análise de
requisitos, a saber

'''
* - Only it can write. Other actors must rely on this to write.
* 
* - Must write, no matter what happens (i.e. write must always
* succeed);
* 
* This means that if for any reason, the media fails to be written to,
* the actor must be able to hold-on information for long enough, or
* provide necessary means to easy the load of current write, until the
* write media goes back to normal;
* 
* - Must, while writing, lock any actor from reading it;
* 
* This is to prevent incomplete/inaccurate info retrieval (i.e. atomic
* write/read operations)
* 
* - Must, while reading, lock any actor from writing it;
* 
* - Must handle parallel writes/reads;
* 
* - Must generate its own events (be able to log itself in event-driven
* manner);
'''

	Dentre os requisitos, destacam se dois importantes: "Must write,
no matter what happens", essencial para a garantia de telemetria dos
sensores, e "must handle parallel writes/reads", conveniente à
implementação de diversos sensores para escrita e leitura simultânea.





% NOTE: Acho que não precisa da biblioteca de compactação, mas seria
% interessante falar que ela foi criada

Para tratar do primeiro requisito, foi desenvolvida uma nova
biblioteca, afim de lidar com o problema de pouca memória presente no
micro, prevendo a possivel falha de todos os canais de comunicacoes do
sistema, por diversas razões.

Para métodos de lidar com dados seriais, pode-se citar fila como sendo
a mais comum. A fila possui um modelo first-in, first out (FIFO) de
consulta, permitindo que n séries s de dados sejam capturados, e para
amostragem ser uniforme, os dados são amostrados numa frequencia f.

No entanto, a lotação da memória por outros processos pode fazer com
que dados sejam perdidos, de forma que alguma série pode não ser
subsequente à outra. Para corrigir isso, uma opção é reduzir a
frequência de captura, por consequência diminuindo sua resolução, em
detrimento da garantia de captura de pelo menos um dado durante o
período em que a memória ficou lotada.

Além disso, também é possivel o agrupamento dos dados em séries, dado
que os dados tem um intervalo bem definido de possilibdades, bastando
apenas guardar os instantes de tempo no qual o mesmo ocorre.

Outro auxílio é possivel através do uso de média movel, para
garantia que os dados, ainda que esparços, sejam capazes de
representar a média de valores de um intervalo.

Por fim, pode-se ainda compactar os valores, procedendo para
escrita numa mídia de armazenamento em massa, ou transmissão para
posterior consulta.

A biblioteca, nomeada de Sparse-Buffer, possui todas as
funcionalidades citadas a cima, utilizada
	
Outro aspecto importante de pensar foi a forma de posicionar os
sensores.

Além disso, a exposicao de eletronicos no mel é um assunto sem
publicações. No entanto, estende-se parte das implicações da presença
de umidade nessas aplicações.

Um problema comum em I2C é sua sensibilidade a efeitos capacitivos
na linha

% https://www.eevblog.com/forum/projects/i2c-pcb-design-trace-length-and-interference/

% descobrir como a resistencia afeta o efeito de bordas na i2c

% Possivel efeito capacitivo do contato de circuito i2c com umidade

% falar sobre hard-fault que engole energia no esp

% falar sobre a necessidade de um debug remoto





[fotografia cabo de rede]

	Para consulta dos sensores, 

I2C



\subsection{Estudos}

\subsubsection{Temperatura}



\subsection{Aplicação e registro}
\subsection{Análise Comparativa}
















