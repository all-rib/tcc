\chapter{Desenvolvimento e Resultados}

O desenvolvimento da plataforma de sensoriamento para colmeias foi
realizado de forma iterativa, em um primeiro momento numa bancada de
trabalho, buscando integrar os sensores, as estratégias de
armazenamento e registro dos dados, levando-se em conta condições
próximas ao uso real, utilizando-se puro mel como material orgânico.
Este capítulo apresenta as etapas de desenvolvimento, os testes
realizados, os principais problemas observados e as soluções propostas
durante o processo.

\section{Arquitetura do Sistema}

A arquitetura proposta para a plataforma de sensoriamento foi
concebida de forma modular, priorizando a facilidade de integração
entre sensores ambientais, unidades de processamento e mecanismos de
armazenamento e comunicação. O objetivo principal é permitir a coleta
contínua de dados.

O sistema é dividido em três camadas: \textbf{Periféricos},
\textbf{Controle}, e \textbf{Energia}, conforme ilustrado na Figura \ref{fig:topologia}.

\begin{figure}[htb]%% Ambiente figure
    %\captionsetup{width=0.55\textwidth}%% Largura da legenda
    \caption{Arquitetura sugerida}%% Legenda
	\label{fig:arquitetura}%% Rótulo
    \includegraphics[scale=2]{arquitetura}%% Dimensões e localização
    \fonte{Autoria Própria}%% Fonte
    \addcontentsline{loge}{figure}{\protect\numberline{\thefigure}Arquitetura Proposta.}
\end{figure}


\begin{itemize}
	\item \textbf{Periféricos:} Composta por sensores de temperatura,
	umidade, pressão e som. Essa configuração permite comparar
	condições internas e externas, formando uma base para a análise do
	comportamento da colônia. . 

	\item \textbf{Controle:} responsável pela leitura dos sensores e
	registro dos dados em mídia de massa. Essa função é desempenhada
	por um microcontrolador ESP32 LORA32 da LilyGO, escolhido por
	integrar conectividade Bluetooth, suporte a cartão SD, e de operar
	múltiplos sensores simultaneamente. Os dados são armazenados em
	formato texto, contendo identificador, carimbo de tempo e unidade
	de medida, garantindo rastreabilidade e simplicidade na análise
	posterior.

	\item \textbf{Energia:} A energia é fornecida
	por baterias LiPo de 11.1V e 5000mAh, reguladas para 5V através de
	regulador de tensão externo.
\end{itemize}

\section{Implementação e Testes Iniciais}

Dentre os sensores escolhidos, o DHT11 é o mais simples, e é comumente
utilizado em projetos hobistas, que apesar de não confiável para
aplicações que exigem precisão e constância, são suficientes para
medidas pontuais de temperatura e umidade. Mais tarde, esse sensor foi
descartado, por sua baixa confiabilidade mesmo nos primeiros testes.

O SHT30, o DS18B20 e o BME280 foram utilizados após problemas de
amostragem com o DHT11, devido à frequência e interferências
provocadas pela interação com matéria orgância. Além disso, o suporte
(como datasheet, source-code do fabricante e notas de aplicação), dá
suporte a verificação de integridade CRC. O DS18B20 é especialmente
interessante, pela possibilidade de contato direto com o mel por não
posuir quaisquer ponto para entradas de material organico.

O controlador possui recomendações de alimentação de 5V pela porta USB
[17]. As [baterias utilizadas](fonte) são do tipo LiPo, de 11.1V e
5000mAh, comuns em projetos de robótica competitiva. Possui 3.7V por
célula, de maneira que duas das células foram utilizadas, totaluzando
7.4V. A sobretensão não apresentou quaisquer problemas no controlador,
seja em testes de processamento ou de consulta de periféricos, com o
único sintoma observado sendo o aquecimento do regulador de tensão
presente na placa.

\begin{photograph}[!htb]%% Ambiente figure
    %\captionsetup{width=0.55\textwidth}%% Largura da legenda
	\caption{Microcontrolador LORA32.}%% Legenda
	\label{fig:lora32}%% Rótulo
    \includegraphics[scale=0.25]{lora32}%% Dimensões e localização
    \fonte{Autoria Própria}%% Fonte
    \addcontentsline{loge}{lora32}{\protect\numberline{\thephotograph}Microcontrolador LORA32.} % Adiciona à lista de ilustrações
\end{photograph}

%[Imagem lora32]

Para integração dos sensores, cabeamento de rede do tipo cat6 foi
utilizado, por sua disponibilidade, excelente condutividade, e
capacidade de redução de efeitos de interferencia, um problema comum
na utilizacao de conectores para a interface I2C[15]. Foi necessario
uma pequena topologia para interligar os sensores à placa, conforme
[imagem X].

Outro ponto a ser citado é a incapacidade da placa em alimentar os
sensores. Foi necessário a utlização de um regulador de tensão
AMS1117.

\begin{figure}[!htb]%% Ambiente figure
    %\captionsetup{width=0.55\textwidth}%% Largura da legenda
	\caption{Topologia do sistema.}%% Legenda
	\label{fig:topologia}%% Rótulo
    \includegraphics[scale=1.5]{topologia}%% Dimensões e localização
    \fonte{Autoria Própria}%% Fonte
    \addcontentsline{loge}{}{\protect\numberline{\thephotograph}Topologia do sistema.} % Adiciona à lista de ilustrações
\end{figure}

Num primeiro experimento, um total de 9 sensores foram conectados,
inicialmente sem qualquer contato com material organico, sendo um
microfone, e, dois de cada um dos sensores a seguir: DHT11, SHT30,
BME280 e DS18B20.


\begin{figure}[htb]
    \caption{Sensores utilizados: (a) BME280, (b) DHT11, (c) DS18B20, (d) SHT30} 
	\label{fig:sensores}
	\centering
	\subfloat[DHT11]{
		\includegraphics[scale=0.7]{dht11}
	}\hspace{0.15cm} 
	\subfloat[SHT30]{
		\includegraphics[scale=0.2]{sht30}
	}
	\newline
	\subfloat[BME280]{
		\includegraphics[scale=0.22]{bme280}
	}
	\subfloat[DS18B20]{
		\includegraphics[scale=0.2]{ds18b20}
	}
	\fonte{Autoria Própria}
    %\addcontentsline{loge}{figure}{\protect\numberline{\thefigure}Telas de cadastro de Paciente: (a) Cadastro Paciente, (b) Cadastro Paciente 2.}
\end{figure}

Os sensores foram conectados em paralelo, permitindo que sejam
consultados sob demanda. Graças à arquitetura I2C, mais de um sensor
do mesmo tipo pode ser utilizado na mesma linha: Os dispositivos são
endereçados de fábrica, e um ou mais bits podem ser alterados
manualmente. O endereçamento é determinado por 7 bits (0b111011x),
sendo os 6 MSB (most-significant-bits) constantes, e o menos
significativo controlado pelo nivel lógico da porta SDO. A conexão
entre SDO e GND força o endereço 0b1110110 (0x76), e a conexão com
VDDIO resulta em 0b1110111 (0x77). O pino não pode flutuar.

Se necessário, mais de dois dispositivos do mesmo tipo podem ser
conectados, aproveitando um detalhe da arquitetura SPI, que utiliza um
terceiro sinal (denominado CS, ou chip-select) para controlar a
seleção de um dispositivo. Neste caso, o sinal SDO tem a mesma função
do CS, mantendo todos os dispositivos com o mesmo endereço (0x76).
Quando for necessário consulta, a linha SDO deve sofrer pull-up,
alterando o endereço para 0x77, destacando-o como tendo um endereço
exclusivo, e portanto responderá individualmente. Ao final da
operação, a linha deve novamente sofrer pull-down, liberando o canal.

Os primeiros problemas se apresentaram no sensor DHT11, que
apresentava leituras constantes e alternadas entre dois valores,
aparenetemente aleatórios, após cada inicialização.

A captura dos dados foram feitas através de consulta sob demanda de
cada sensor, utilizando-se em sua grande maioria o código fonte
fornecido pelo fabriante, e gravação dos dados ocorreu em
armazenamento utilizando uma função atômica de escrita, em um arquivo
de texto, conforme o padrão a seguir

\texttt{[#n][Timestamp][Nome]([medida][unidade]}

Esse padrão foi utilizado por fornecer contexto de origem do dado,
instante no qual foi capturado, sequencia entre várias amostrasa, e
respectiva medida de cada sensor, consideradas suficientes para a
naturza desta aplicação.

Finalmente, o microcontrolador, bem como uma bateria de alimentação,
foram isolados entre sí, e inseridos em frascos de conserva, selando
pontos de passagem de cabos com resina do tipo epoxi, evitando
exposição à umidade. Um conector foi deixado do lado de fora, para
facilitar a troca de bateria, caso necessário.

Após isso, os sensores foram testados mergulhando os mesmos por
inteiro dentro do mel. Os resultados podem ser observados na  [figura
](figura comparativa dentro e fora do mel). Nos testes executados, o
sistema rodou sem problema por três dias ininterruptos.

\section{Integração em Campo}

Após os testes de bancada, o sistema foi instalado em colmeias de
\textit{T. angustula}, com sensores posicionados tanto no
interior quanto no exterior das caixas, de forma a permitir
comparações entre as condições ambientais internas e externas.

Para integração com a colméia, o controlador e bateria foram presos ao
pilar de suporte da colméia, conforme [figura X](inserir foto da
colmeia), e os sensores dispostos no sobreninho mais superior, distante
do contato com os favos, prevendo possiveis problemas de funcionamento
dos sensores devido exposição ao mel e a cera.

Foi necessário selar quaisquer pontos de conexão e passagem de fios
dentro e no limiar da colméia. Isso foi necessário devido à atividade
das abelhas de selar quaisquer buracos, e esterelizar materiais
internos, cobrindo-os com uma liga de cera e própolis. Esse processo
acontece devido ao ataque de outros insetos, em especial formigas e
fungos, que se aproveitam dos pontos de abertura que os cabos
provocam. Além disso, corpos estranhos também podem estar infectados,
ou possuir material noscivo às abelhas. Essa prática não danifica os
sensores, mas impediram a comunicação perfeita do sensor, graças à
alterações na radiação térmica, leitura absoluta dos dados, e
alterando o meio de propagação de ondas sonoras. Poucos dados foram
capturados, tendo em vista as interferências observadas.

Para o ponto de passagem dos cabos, no entanto, as abelhas selaram a
entrada dos fios, criando um canal justo, e facilmente renovável por
elas, além de ser familiar para os indivíduos da colônia, e portanto
benéfica.

\section{Problemas Observados}

Surgiram falhas recorrentes no barramento I²C, causando travamentos e
reinicializações automáticas do sistema.  Tais eventos geravam
acionamentos constantes do \textit{watchdog}, o que, embora evitasse
travamentos permanentes, resultava em alto consumo energético e
drenagem prematura da bateria.

Os sensores sofriam de constantes falhas nos sinais e conexões. A
suposição é de que o ambiente interno da colméia possui uma interação
distinta quando comparada ao sensor em contato com o mel. Além disso,
a utilização de resinas ou outros produtos para impermeabilizar estes
sensores interferem na sua precisão, e na saúde dos insetos, pois em
sua grande parte, eles possuem solventes e outros materiais tóxicos.

Para observação dos eventos de falha, o uso da API padrão dos
dispositivos ESP32 se torna exaustiva do ponto de vista de
desenvolvimento. As APIs possuem chamadas para geração de logs, capaz
de apresentar mensagens conforme a demanda do usuário. Conforme
explica [18], existem diversos níveis de log, onde, ao configurar
alguns parâmetros, podemos determinar quais mensagens aparecerão. Mas
a quantidade de eventos gerados era tamanha que a manutenção (seja em
remover, adicionar, ou alterar) torna o processo impraticável.

Os eventos gerados também geravam concorrência no processo de escrita
no cartão. Além do sistema prosseguir tentando capturar os dados em
sensores com erros, mesmo a falha em escrever os dados provocavam mais
eventos, e por consequência, as rotinas paravam por completo.

Por vezes ainda, watchdogs entravam em ação, e o processo de reset
drenava a bateria completamente. Isso ocorre em decorrência do alto
processamento inicial para inicializar o dispositivo e entrar em
regime de funcionamento. Os sensores, mesmo com suas funcionalidades
de economia de energia, quando precisavam ser inicializados a cada
nova

Finalmente, o diagnostico do sistema em campo é difícil, pela
impossibilidade de debug, uso de osciloscópio, ou
observação dos eventos em tempo real. As interferências nos sensores
provocaram erros do tipo hard-fault (por padrão esta rotina possui um loop
infinito). A intervenção do usuário, então, sempre seria necessaria.

\section{Soluções Propostas}

Os problemas apresentados podem ser separados em 3 categorias:
 - falhas de hardware, em que um sensor possui falhas intermitentes;
 - Concorrência, onde a quantidade de eventos geradas preenche a memória;
 - Resets, onde a perca de informações ocorre quando o sistema
   reinicia;

Para as falhas de hardware, mais testes em campo são necessários, uma
vez que a interação entre eletrônicos, mel e materiais orgânicos não
possui registros em literatura. Assim sendo, a exclusão do contato
entre a colmeia e o sistema é essencial.

Duas soluções são apresentadas:

- Adaptar o circuito, de forma que os sensores sejam integrados à uma
estruturas de encapsulamento e vedação, promovendo a exclusão do
contato entre mel e circuito, sem o uso de substâncias noscivas. Essa
opção tem a vantagem de que alguns sensores possuem mais de um
encapsulamento, o que pode facilitar na confecção de containeres para
os mesmos. Nesta obra, uma nova placa integrada de sensores foi
elaborada, \textbf{Aristeu}, com a intenção de utilizar silicone
alimentar para isolar os circuitos, mas não foi posta em produção,
devido ao custo e tempo do projeto. O projeto encontra-se disponível
repositório próprio e público;

- Adaptar a caixa, permitindo um posicionamento planejado dos
sensores de maneira independente, entalhando canais e sulcos para
posicionamento. Essa solução é mais barata, com a desvantagem de
lenta produção, uma vez que depende do cresimento da
colméia. Além disso, por ser uma adaptação do
padrão INPA, o novo modelo pode ser impraticável quando escalado,
considerando as adaptações necessárias.

Para contornar as limitações de geração de eventos, foi
desenvolvida a biblioteca \textbf{Hermes}, concebida como um sistema
de registro de eventos em tempo real, com foco em confiabilidade e
isolamento de escrita. Cada interação entre o microcontrolador e os
sensores gera um ou mais eventos categorizados em "chains": inicialização, configuração,
consulta ou desinicialização; Isso permite rastrear as ações de forma
estruturada. A biblioteca implementa atomicidade de leitura e escrita
e bloqueio mútuo de acesso aos arquivos de log, evitando corrupção de
dados durante operações paralelas.

Em efeitos práticos, a biblioteca é um wrapper, que automatiza a
geração das mensagens ao criar os conceicos de \textbf{object}, que
executa \textbf{chains}. Um objeto pode ser representado por
periférico, que possui diversas rotinas a serem executadas:
inicializar, configurar, amostrar e desinicializar. Essas rotinas
possuem uma cadeia interna de funções que executam, separadamente,
passos para realizar uma determinada tarefa. Hermes abstrai a
complexidade utilizando ponteiros de funções para abstrair as cadeias
originais da função, e insere eventos informativos entre cada uma das
chamadas. Os eventos então são gerados apenas por um ator, e escritos
no cartão sem concorrência.

Além disso, a biblioteca foi implementada com uma análise de
requisitos, a saber

\begin{verbatim}

- Only it can write. Other actors must rely on this to write.

- Must write, no matter what happens (i.e. write must always succeed);

This means that if for any reason, the media fails to be written to,
the actor must be able to hold-on information for long enough, or
provide necessary means to easy the load of current write, until the
write media goes back to normal;

- Must, while writing, lock any actor from reading it;

This is to prevent incomplete/inaccurate info retrieval (i.e. atomic
write/read operations)

- Must, while reading, lock any actor from writing it;

- Must handle parallel writes/reads;

- Must generate its own events (be able to log itself in event-driven
manner);

\end{verbatim}

	Dentre os requisitos, destacam se dois importantes: "Must write,
no matter what happens", essencial para a garantia de telemetria dos
sensores, e "must handle parallel writes/reads", conveniente à
implementação de diversos sensores para escrita e leitura simultânea.

A arquitetura da biblioteca é apresentada na [figura X]

\begin{figure}[!htb]%% Ambiente figure
    %\captionsetup{width=0.55\textwidth}%% Largura da legenda
	\caption{Arquitetura da biblioteca \textbf{HERMES}.}%% Legenda
	\label{fig:hermes}%% Rótulo
    \includegraphics[scale=0.6]{hermes}%% Dimensões e localização
    \fonte{Autoria Própria}%% Fonte
    %\addcontentsline{loge}{}{\protect\numberline{\thephotograph}Arquitetura da biblioteca \textbf{HERMES}.} % Adiciona à lista de ilustrações
\end{figure}

Complementarmente, e de encontro ao último problema, foi criada a
biblioteca \textbf{Sparse-Buffer}. Esta foi projetada para lidar com
restrições de memória, e possíveis concorrências de escrit para
grandes amostras de dados. Essa ferramenta emprega filas do tipo
\textit{First-In, First-Out} (FIFO) e técnicas de média móvel e
compactação de dados, reduzindo a quantidade de escritas durante
períodos de instabilidade. Quando a mídia de gravação está
indisponível, os dados são retidos temporariamente em memória, e
compactados até que a operação possa ser concluída de forma segura.

Enquanto isso, dois processos simples são aplicados:

- Novas amostras são agrupadas em séries para evitar a repetição: Dado
um dado que varie num intervalo determinado (como a temperatura), os
dados podem ser agrupados em amostras que se repetem em diferentes
medidas de tempo, e no melhor caso, reduzindo o espaço de 2N, sendo N
o número de amostras, para N+1, de forma que a medida identifica a
série, e os valores são os instantes de tempo no qual a medida
ocorreu. 

- A resolução de captura dos dados é reduzida: Ao preencher a fila de
dados, um a cada N dados podem ser descartados, de forma que a
resolução final da observação é reduzida, mas a média dos valores se
mantém para grandes períodos de captura;

As rotinas podem ser observadas em \autoref{alg:alg1} e \autoref{alg:alg2}.


%\begin{sourcecode}[htb]
%\caption{\label{codigo:classeFoo}Classe Aluno}
%\begin{lstlisting}[frame=single, language=Java]
%@Entity
%public class Foo {
%
%    @Id
%    @GeneratedValue(strategy = GenerationType.IDENTITY)
%    private Long id;
%
%    private String nome;
%
%    private Integer ra;
%
%    // constructor, getters and setters
%}
%\end{lstlisting}
%\fonte{}
%\end{sourcecode}

\begin{algorithm}[htb]
\caption{Agrupamento de novas amostras em séries}
\label{alg:alg1}%% Rótulo
\hrule
\begin{algorithmic}[1]
\Require Fluxo de (valor, tempo); intervalo\_determinado
\Ensure Conjunto de séries com valor representativo e lista de tempos
\ForAll{nova\_amostra \textbf{em} fluxo\_de\_dados}
	\If{nova\_amostra.valor \textbf{em} intervalo\_determinado}
		\State procurar \textit{série\_existente} com \textit{série.valor} $\approx$ nova\_amostra.valor
		\If{\textit{série\_existente} encontrada}
			\State adicionar nova\_amostra.tempo em \textit{série\_existente.lista\_tempos}
		\Else
			\State criar \textit{série} com \textit{valor} $\gets$ nova\_amostra.valor
			\State \textit{série.lista\_tempos} $\gets$ [\,nova\_amostra.tempo\,]
			\State adicionar \textit{série} ao \textit{conjunto\_de\_séries}
		\EndIf
	\EndIf
\EndFor
\end{algorithmic}
\hrule
\fonte{Autoria própria}%% Fonte
\end{algorithm}

\begin{algorithm}[htb]
\caption{Redução da resolução (subamostragem 1 em N)}
\label{alg:alg2}%% Rótulo
\hrule
\begin{algorithmic}[2]
\Require fila\_original, fator $N \ge 2$
\Ensure fila\_reduzida
\State contador $\gets 0$
\State fila\_reduzida $\gets$ lista\_vazia
\ForAll{dado \textbf{em} fila\_original}
	\State contador $\gets$ contador $+ 1$
	\If{contador \% N $==$ 0}
		\State adicionar dado à fila\_reduzida
	\EndIf
\EndFor
\end{algorithmic}
\hrule
\fonte{Autoria própria}%% Fonte
\end{algorithm}

\section{Discussão}

A aplicação prática do sistema em colmeias reais revelou desafios
raramente documentados na literatura sobre apicultura de precisão. A
maioria dos trabalhos correlatos foca em modelos teóricos de
sensoriamento, mas não descreve as dificuldades inerentes à exposição
dos sensores em ambientes biológicos ativos, sujeitos a variações
térmicas, alta umidade e ação direta dos insetos.

Os resultados indicam que sensores eletrônicos convencionais exigem
adaptações específicas de encapsulamento e isolamento elétrico para
operar de forma confiável dentro de colmeias. As falhas observadas
reforçam a necessidade de pesquisas voltadas à durabilidade de
componentes em ambientes orgânicos, bem como ao desenvolvimento de
metodologias de calibração e filtragem que considerem interferências
biológicas e físicas.

Ainda que os testes tenham se limitado a períodos curtos, o estudo
forneceu insights relevantes sobre a integração de sistemas embarcados
em contextos naturais complexos. As bibliotecas e soluções propostas
representam avanços significativos na robustez e autonomia de
plataformas de monitoramento ambiental, servindo de base para futuras
iterações com novos sensores, protocolos e estratégias de energia.

%
%
%
%
%
%
%
%
%
%
%
%
%
%
%
%
%
%
%
%
%
%
%
%
%
%
%
%% TODO: Deveria usar a ideia do Denardim, e criar uma região reservada
%% para registro de logs?
%
%% TODO: Ver como funciona o debug de um hard-fault dentro de um RTOS
%
%
%% NOTE: Acho que não precisa da biblioteca de compactação, mas seria
%% interessante falar que ela foi criada
%
%Para tratar do primeiro requisito, foi desenvolvida uma nova
%biblioteca, afim de lidar com o problema de pouca memória presente no
%micro, prevendo a possivel falha de todos os canais de comunicacoes do
%sistema, por diversas razões.
%
%Para métodos de lidar com dados seriais, pode-se citar fila como sendo
%a mais comum. A fila possui um modelo first-in, first out (FIFO) de
%consulta, permitindo que n séries s de dados sejam capturados, e para
%amostragem ser uniforme, os dados são amostrados numa frequencia f.
%
%No entanto, a lotação da memória por outros processos pode fazer com
%que dados sejam perdidos, de forma que alguma série pode não ser
%subsequente à outra. Para corrigir isso, uma opção é reduzir a
%frequência de captura, por consequência diminuindo sua resolução, em
%detrimento da garantia de captura de pelo menos um dado durante o
%período em que a memória ficou lotada.
%
%Além disso, também é possivel o agrupamento dos dados em séries, dado
%que os dados tem um intervalo bem definido de possilibdades, bastando
%apenas guardar os instantes de tempo no qual o mesmo ocorre.
%
%Outro auxílio é possivel através do uso de média movel, para
%garantia que os dados, ainda que esparços, sejam capazes de
%representar a média de valores de um intervalo.
%
%Por fim, pode-se ainda compactar os valores, procedendo para
%escrita numa mídia de armazenamento em massa, ou transmissão para
%posterior consulta.
%
%A biblioteca, nomeada de Sparse-Buffer, possui todas as
%funcionalidades citadas a cima, utilizada
%	
%Outro aspecto importante de pensar foi a forma de posicionar os
%sensores.
%
%Além disso, a exposicao de eletronicos no mel é um assunto sem
%publicações. No entanto, estende-se parte das implicações da presença
%de umidade nessas aplicações.
%
%Um problema comum em I2C é sua sensibilidade a efeitos capacitivos
%na linha
%
%% https://www.eevblog.com/forum/projects/i2c-pcb-design-trace-length-and-interference/
%
%% descobrir como a resistencia afeta o efeito de bordas na i2c
%
%% Possivel efeito capacitivo do contato de circuito i2c com umidade
%
%% falar sobre hard-fault que engole energia no esp
%
%% falar sobre a necessidade de um debug remoto
%
%
%
%
%
%[fotografia cabo de rede]
%
%	Para consulta dos sensores, 
%
%I2C
%
%
%
%\subsection{Estudos}
%
%\subsubsection{Temperatura}
%
%
%
%\subsection{Aplicação e registro}
%\subsection{Análise Comparativa}
%
%
%
%
%
%
%
%
%
%
%
%
%
%
%
%
