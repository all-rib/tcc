\section{Abelhas, apicultura e apicultura de precisão}

As abelhas são insetos alados (clado Anthophila) responsáveis pelo processo da
polinização e produção de mel. As mais de 20 mil espécies, presentes em
todos os continentes com exceção da Antártida, se alimentan de pólen
e nectar, dando origem ao processo de polinização, impotante não
apenas para as plantas, mas para todo o ecossistema da produção de
alimentos, e principalmente, para a apicultura, atividade existente
desde o Egito Antigo, e de crescente importância no Brasil.

No Brasil, as espécies mais comuns compreendem a Apis mellifera
(Africanizada), Tetragonisca angustula (Jataí), e Eulaema nigrita
(Abelha das Orquídeas)[8], sendo aquela a mais numerosa, dado o
interesse do mercado brasileiro em sua alta produtividade para o
mercado apícola[10].

A domesticação do processo de produção de mel recebe o nome de
apicultura. É um processo milenar, praticado deste o Egito antigo, que
consiste em alocar as abelhas em uma região arbitária, proporcionar à
colmeia condições de crescimento (disponibilidade de alimento, água, e
condições climáticas adequadas), e extrair o mel periódicamente,
administrando também a criação de outras colméias, e competição entre
espécies.

É importante destacar que a apicultura geralmente se refere à produção
de mel num panorama geral, mas também pode significar a produção de
mel exclusivamente por abelhas do genero Apis (abelhas com ferrão), ou
tribo Apini, sendo sua maior representante a A. mellifera. Surge então
a meliponicultura, que trata da produção pela tribo Meliponini
(popularmente chamadas de abelhas sem ferrão). Para facilidade de
entendimento, tomar-se-à apicultura geral (APG) como sendo todo o processo
apícola, indistinto de espécies, e apicultura específica (APE), ou
simplesmente apicultuta como a produção de mel por abelhas do genero
Apis.

A APG teve início no Brasil na década de 1830, através da
importações de espécimes da Europa, a pedido do Pe. Antonio Pinto
Carneiro. A prática não era rentável, além de perigosa, e por algum
tempo foi considerada inviável, sendo retomada apenas em 1970, graças
a esforços de pesquisa apresentados no Congresso Brasileiro de
Apicultura, em Florianópolis.[5]

Cerca de 40 anos se passaram até que os métodos e tecnologias da
agricultura de precisão (AGP) fizessem luz à ao processo produtivo do
mel, dando origem então à apicultura de precisão (APP). A aplicação de
técnicas análogas de sensoriamento e controle comprovadamente otimizou
a produção de mel de diversas maneiras, entre impacto no ambiente, nas
espécies, quantidade de produtos (mel e própolis) e subprodutos
produzida (cera, medicamentos e cosméticos).

Pode-se definir a APP considerando-se alguns princípios da AGP: essa
trata da aquisição, processamento e análise de dados para permitir
otimizações no gerenciamento e na tomada de decisões do processo
agrícola. De maneira análoga, os mesmos estudos podem ser inseridos na
apicultura, se tornando então um processo de aquisição, processamento
e análise de dados afim de otimizar o gerenciamento e a tomada de
decisões no processo apícola.

\subsection{Tetragonisca angustula}

A Tetragonisca angustula, popularmente conhecida como jataí, é uma
abelha social da tribo Meliponini, amplamente distribuída nas regiões
tropicais e subtropicais da América Latina, incluindo o Brasil.
Trata-se de uma das espécies de abelhas sem ferrão mais estudadas e
manejadas na meliponicultura, destacando-se por sua docilidade,
adaptabilidade e importância ecológica como polinizadora generalista .

Morfologicamente, a jataí é uma abelha de pequeno porte, medindo entre
4 a 5 mm de comprimento. Apresenta cabeça e tórax de coloração escura,
com abdômen mais claro e pernas pardacentas. Uma característica
distintiva é a ausência de ferrão funcional, comum a todas as abelhas
da tribo Meliponini, o que a torna incapaz de ferroar. Suas asas
possuem venação reduzida, e as operárias apresentam cerdas nas pernas
posteriores, adaptadas para o transporte de pólen .

[imagem jatai]

Apesar de sua ampla distribuição e relevância ecológica, T. angustula
ainda carece de uma base técnica consolidada na literatura científica,
especialmente no que diz respeito a parâmetros biométricos
padronizados, dados sobre produtividade, comportamento reprodutivo, e
manejo racional.

A organização social da colônia é altamente complexa, composta por
várias castas, incluindo rainha, operárias e zangões. As colônias são perenes, podendo conter de
20 a 80 mil indivíduos, abrigadas geralmente em cavidades protegidas.

\subsection {Colmeias jatai}

A criação de ambas abelhas exigem a utilização de caixas pré-montadas
que permitem a modularização da coleta de mel. Com modelos diferentes
para cada uma das especies, o modelo ideal para as abelhas-sem-ferrão
é denominado modelo INPA.

As caixas modelo INPA foram desenvolvidas no Brasil para a criação de
abelhas nativas. O nome “INPA” refere-se ao Instituto Nacional de
Pesquisas da Amazônia, onde esse modelo começou a ser utilizado e
disseminado, especialmente nas décadas de 1980 e 1990, como parte dos
esforços para promover a meliponicultura racional, ecológica e
adaptada às espécies nativas.

Diferentemente das caixas utilizadas na apicultura com Apis mellifera,
que adotam quadros verticais móveis, o modelo INPA respeita a
arquitetura natural dos ninhos das abelhas sem ferrão, que constroem
discos horizontais de cria empilhados, cercados por potes de mel e
pólen em estruturas orgânicas. As caixas INPA reproduzem essas
condições em um ambiente controlado e modular, permitindo o manejo
sustentável sem desorganizar o ninho.

Esta é composta por módulos empilháveis, cada um com função distinta e
dimensões adaptadas ao porte da espécie criada. A estrutura básica
inclui:

Módulo de ninho: Onde são depositados os ovos e construídos os discos
de cria. Costuma medir cerca de 15x15x9, considerando-se as paredes
internas, com cerca de 3cm de espessura. O espaço interno é fixo, sem
divisórias móveis, permitindo que as abelhas organizem o ninho de
forma natural.

Sobreninho (ou extensão do ninho): Módulo adicional posicionado acima
do ninho. Permite o crescimento da área de postura e armazenamento,
acomodando a expansão natural da colônia com o tempo.

Melgueira: Espaço dedicado ao armazenamento de mel. Nela as abelhas
constroem pequenos potes de cera onde depositam o mel. A extração é
feita manualmente, com seringas ou espátulas, respeitando a
integridade da colônia.

Fundo (com ou sem lixeira): Parte inferior da caixa. Pode ter um
espaço adicional para acumular resíduos (a chamada "lixeira"), que
facilita a limpeza periódica e o controle de pragas.

Tampa e cobertura externa: Fecham a caixa superiormente, ajudando a
manter o microclima interno, protegendo contra luz, calor excessivo e
umidade.

[Fotografia colmeia jatai INPA]

\Section{Sesnores}

Sensores são dispositivos capazes de detectar mudanças no ambiente, e
gerar eventos sobre estas mudanças para um sistema. No meio apícola,
sensores tem ampla aplicação afim de detectar a maioria dos eventos
pertinentes para a colméia,

O processo aquisitório de dados pode incluir métodos invasivos (como a
inserção de sensores diretamente dentro da colméia, captura de
indivíduos, controle de entrada e saída, e afins), ou não invasivos
(monitoramento de temperatura externa, umidade do ar, peso, entre
outras). Métodos invasivos costumam ter resultados mais precisos, com
o contraponto de que podem interferir diretamente na acurácia dos
dados, em decorrência das alterações que o procedimento provoca no
ambiente, interferindo assim no comportamento natural dos indivíduos.
Métodos não invasivos costumam ser mais acurados, ao custo da
precisão, tendo em mente o espectro de desvios que o sensoriamento
sofre em decorrência de interferências do ambiente: vento, exposição
ao sol, esparcidade dos insetos, etc.

Dentre a enorme gama de aplicações, quatro análises se destacam:
sensoriamento de temperatura, umidade, peso e som.

Entre os muitos, cita-se especialmente BME280, DHT11, DS18B20, SHT30,
INMP441. Esses sensores cobrem a maior parte do escopo deste projeto,
a saber:
o

O BME280 é um sensor Bosch que combina sesnores digitais de umidade,
temperatura, e pressão. Possui um encaplulamento LGA compacto de
2.5x2.5mm², e espessura de 0.93mm.

O DHT11 é um sensor de umidade e temperatura, digital, de baixo custo,
mas atualmente baixa confiabilidade. é Utilizado na maioria dos
projetos hobbystas, devido sua disponibilidade. O fornecedor original
é desconhecido, mas a sua grande maioria é importada sob a marca
AOSONG.

O DS18B20 é um sensor digital de temperatura fabricado pela Maxim
Integrated (atualmente parte da Texas Instruments), amplamente
utilizado em sistemas embarcados e aplicações industriais. Seu
destaque está na comunicação via 1-Wire, que permite a transmissão de
dados e alimentação por um único fio, simplificando o cabeamento em
sistemas distribuídos. O sensor oferece uma faixa de medição de -55 °C
a +125 °C, com precisão típica de ±0,5 °C na faixa entre -10 °C e
+85 °C, e resolução configurável de 9 a 12 bits. Além disso, cada
unidade possui um código único de 64 bits gravado de fábrica,
permitindo a identificação individual de múltiplos sensores conectados
em paralelo no mesmo barramento. Seu encapsulamento compacto (TO-92 ou
versões estanques em aço inox) e sua robustez tornam o DS18B20 ideal
para uso em ambientes externos, automação residencial, agricultura de
precisão e sistemas de monitoramento térmico distribuído.

O SHT30, desenvolvido pela Sensirion, é um
sensor digital de temperatura e umidade relativa, ideal para
aplicações em sistemas embarcados. Ele utiliza comunicação via I²C com
suporte a verificação de integridade CRC, o que o torna confiável
mesmo em ambientes com interferência elétrica. Possui calibração de
fábrica, o que elimina a necessidade de ajustes manuais, e oferece
alta precisão, com erro típico de ±2\% para umidade relativa e ±0,3 °C
para temperatura. Com baixo consumo de energia e encapsulamento
compacto (DFN-6), o mesmo é amplamente utilizado em aplicações
industriais, automação residencial, dispositivos IoT e monitoramento
ambiental. Sua combinação de precisão, estabilidade e facilidade de
integração o torna uma escolha robusta e eficiente para projetos que
exigem sensoriamento ambiental confiável.

Escolhidos pela disponibilidade e precisao, alem de possuírem um
sensor de umidade embarcado. seu tambanho também facilita o
posicitionamento em qualquer local da colmeia. O sensor possui
interface I2C, com velocidades de ate 1MHz, com acuracia de 1.5\% de
umidade relativa, e 0.1oc. Seu baixo consumo (1.5mA durante medicao, e
0.5uA em idle-mode) tambem e um fator importante para a aplicacao,
considerando que o dispositivo deve rodar sem supervisao durante um
grande periodo, utilizando baterias. Por fim, o dispositivo nao
necessita de calibracao.

O INMP441 é um microfone digital do tipo MEMS
(Micro-Electro-Mechanical Systems), projetado para capturar sinais
acústicos com alta fidelidade e baixo ruído. Ele integra internamente
um conversor ADC de 24 bits e se comunica por meio da interface
digital I²S, permitindo a transmissão direta de áudio para
microcontroladores ou processadores de sinal sem a necessidade de
circuitos analógicos externos. O INMP441 apresenta uma faixa dinâmica
ampla e resposta em frequência plana, sendo capaz de captar sons entre
60 Hz e 15 kHz com boa sensibilidade. Sua arquitetura digital elimina
interferências comuns a microfones analógicos, tornando-o ideal para
aplicações como reconhecimento de voz, gravação de áudio embarcada,
sistemas de monitoramento acústico e dispositivos IoT com entrada
sonora. Com encapsulamento compacto e fácil integração em placas de
circuito impresso, o INMP441 oferece uma solução prática e robusta
para projetos embarcados que exigem captura de áudio de forma precisa
e eficiente.

\Section{I2C}

O I²C (Inter-Integrated Circuit), também conhecido como I2C, é um
protocolo de comunicação serial síncrono utilizado em sistemas
embarcados para a troca de dados entre microcontroladores e
dispositivos periféricos. Desenvolvido originalmente pela Philips
Semiconductor na década de 1980, foi projetado para permitir a
comunicação de múltiplos dispositivos utilizando apenas duas linhas de
sinal, além da alimentação: uma linha de dados (SDA – Serial Data) e
outra de clock (SCL – Serial Clock).

[Diagrama de tempo do I2C]



% TODO: Rever abaixo o que jogar fora e o que manter



%A ideia de sensoriar as colméias diretamente não é nova. Alguns
%trabalhos demonstram que a ideia intriga pesquisadores à cerca de 20
%anos.[?] No Brasil, apesar de recente, a ideia também tem se difundido
%bastante, com o surgimento de pesquisas e soluções comerciais para
%auxiliar a tomada de decisões.[?]
%
%A temperatura, bem como a umidade, como fator decisivo para o ritmo de
%trabalho de todas as espécies, possui adoção quase que unanime entre
%outros trabalhos.[?] A facilidade de implantação dos sensores,
%processamento, e intepretação, permite extrair uma relação direta
%entre a termorregulação da colmeia, ciclo de trabalho e eficiência
%produtiva.[?]
%
%Sua aplicação, no entanto, não tem uma discução direta em se tratando
%da metodolgia. Simples variáveis, como por exemplo o posicionamento
%dos sensores, podem alterar drasticamente as medidas. E os tipos de
%interferências possíveis também pode ser um fator crucial.
%
%O som, advindo do comportamento de enxame dessas criaturas, também
%descreve um espectro de comportamento. É sabido que o som composto
%pela colméia reflete tanto o estado atual, seja este fome, doença,
%ameaça, quanto uma breve previsão de eventos, como enxameação, mudança
%de ritmo produtivo, ou mesmo o compartilhamento de informações,
%através do comportamento de "dança".[?]
%
%Aplicações práticas demonstram a possibilidade de induzir certos
%comportamentos, condicionando enxames inteiros à serem dóceis,
%permitindo assim a coleta de mel sem a presença de riscos, sejam para
%os coletores, quanto para a própria abelha.[?]
%
%De maneira semelhante, os mesmos estudos no entanto, não analisam o
%espectro das posiçções possíveis, bem como da diferença de som entre
%as diversas espécies, ou o tipo de microfone. é conhecido que
%diferentes espécies geram diferentes espectros de ruídos, de forma que
%pode haver mais de uma forma de analizar determinada espécie através
%do áudio.
%
%E por fim, o peso evidencia a evolução da produção de alimentos e
%indivíduos dentro da colmeia. Através da sua medida, o apicultor
%consegue entender o ritmo de produção e consumo da colméia[12],
%habilitando ações como alimentação artificial em épocas de escacez, ou
%mesmo a melhor época para colheitas.
%
%Extração de audio pode automatizar a identificação de fatores
%de stress. Para isso é necessário analizar os audios sistematicamente.
%Existem modelos[Kulyukin, 2018] de redes neurais capazes classificar
%sons de abelhas dentre outros sons. O projeto faz uso de sensores como
%microfone, temperatura e câmera para tratar diferenciaçãode abelhas de
%outros insetos, detecção e previsão de eventos na colmeia, detecção de
%doenças, e associação de eventos à determinadas condições de produção
%
%	Utilizando-se de caixas no padrão Langstroth, o estudo foi capaz
