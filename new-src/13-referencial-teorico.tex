 - O que é uma abelha?
 - O que é uma colméia?
 - O que é um apiário?

 - O que é monitoramento?
 - Porque o monitoramento funciona?

 - No que consistem os sensores?











Abelhas, apicultura e apicultura de precisão

As abelhas são insetos alados (clado Anthophila) responsáveis pelo processo da
polinização e produção de mel. As mais de 20 mil espécies, presentes em
todos os continentes com exceção da Antártida, se alimentan de pólen
e nectar, dando origem ao processo de polinização, impotante não
apenas para as plantas, mas para todo o ecossistema da produção de
alimentos, e principalmente, para a apicultura, atividade existente
desde o Egito Antigo, e de crescente importância no Brasil.

No Brasil, as espécies mais comuns compreendem a Apis mellifera
(Africanizada), Tetragonisca angustula (Jataí), e Eulaema nigrita
(Abelha das Orquídeas)[8], sendo aquela a mais numerosa, dado o
interesse do mercado brasileiro em sua alta produtividade para o
mercado apícola[10].

A domesticação do processo de produção de mel recebe o nome de
apicultura. É um processo milenar, praticado deste o
Egito antigo, que consiste em alocar as abelhas em uma região
arbitária, proporcionar à colmeia condições de crescimento (disponibilidade de alimento, água, e condições climáticas adequadas), e extrair o mel periódicamente, administrando também a criação de outras colméias, e competição entre espécies.

É importante destacar que a apicultura geralmente se refere à produção
de mel num panorama geral, mas também pode significar a produção de
mel exclusivamente por abelhas do genero Apis (abelhas com ferrão), ou
tribo Apini, sendo sua maior representante a A. mellifera. Surge então
a meliponicultura, que trata da produção pela tribo Meliponini
(popularmente chamadas de abelhas sem ferrão). Para facilidade de
entendimento, tomar-se-à apicultura geral (APG) como sendo todo o processo
apícola, indistinto de espécies, e apicultura específica (APE), ou
simplesmente apicultuta como a produção de mel por abelhas do genero
Apis.

A APG teve início no Brasil na década de 1830, através da
importações de espécimes da Europa, a pedido do Pe. Antonio Pinto
Carneiro. A prática não era rentável, além de perigosa, e por algum
tempo foi considerada inviável, sendo retomada apenas em 1970, graças
a esforços de pesquisa apresentados no Congresso Brasileiro de
Apicultura, em Florianópolis.[5]

Cerca de 40 anos se passaram até que os métodos e tecnologias da
agricultura de precisão (AGP) fizessem luz à ao processo produtivo do
mel, dando origem então à apicultura de precisão (APP). A aplicação de
técnicas análogas de sensoriamento e controle comprovadamente otimizou
a produção de mel de diversas maneiras, entre impacto no ambiente, nas
espécies, quantidade de produtos (mel e própolis) e subprodutos
produzida (cera, medicamentos e cosméticos).

Pode-se definir a APP considerando-se alguns princípios da AGP: essa
trata da aquisição, processamento e análise de dados para permitir
otimizações no gerenciamento e na tomada de decisões do processo
agrícola. De maneira análoga, os mesmos estudos podem ser inseridos na
apicultura, se tornando então um processo de aquisição, processamento
e análise de dados afim de otimizar o gerenciamento e a tomada de
decisões no processo apícola.

Técnicas rudimentares para otimização produtiva

O produção de mel para as abelhas é quase um todo da sua
sobrevivência, tendo em vista sua função alimentar para os indivíduos,
e suas crias.  Idealmente, elas não cessam a produção de mel, e
enquanto houver disponibilidade de recursos, e espaço para novas
crias, a colméia seguirá crescendo.

Mas na realidade, aspectos como estação do ano, clima, doenças, e
competição com outras espécies, acabam por determinar um passo de
cresimento com limites bem definidos[11], e portanto, o controle sobre
esses aspectos ditam o ápice produtivo da colmeia.

A observação dessas variáveis levou o homem à desenvolver tecnicas e
tenologias para o controle. Históricamente, isso se deu principalmente
a partir do século 19. Fatores como espaço, exposição ao clima, e
disponibilidade de alimento são elementos facilmente controláveis pelo
produtor, e seu estudo é mister para o processo produtivo.

Como um passo inicial, observa-se a padronização da colmeia. Concebida
por Lorenzo L. Langstroth, em 1851, os padrões de medida da caixa
otimizaram o cresimento e manutenção das colméias, tendo em vista a
densidade de produção (relação entre espaço livre, e espaço produtivo
necessário) natural que as abelhas exigem[6]. Esse padrão, geralmente
implementado para crias de espécie \it{Apis mellifera} (abelha-europa,
ou abelha-comum) se adequa à uma separação natural de funções que as
abelhas realizam: a parte inferior da colméia comporta postura e
crias, enquanto as superiores servem para alimentação e reservas.
Então o crescimento da colméia pode ser mais facilmente controlado e
monitorado pelo apicultor.

A arquitetura de Langstroth também altera a forma natural com que as
abelhas constroem os favos. Naturalmente, eles são curvados, de
maneira que a extração do mel é extremamente danosa para a estrutura
da colméia em sí.  Já dentro dos apiários, de maneira estruturada, as
caixas possuem quadros sobre os quais as abelhas podem construir
paredes retas e de dimensões bem definidas, o que facilita a
colheita[6].

Outra variável é a exposição solar. Muita expoxisão ao sol significa
baixa umidade, e nível maior de calor, dificultando a sobrevivência no
verão. Ao passo que pouca significará dificuldade de orientação, e
temperaturas mais baixas, dificultando a sobrevivência no inverno.
Durante as épocas de abertura das flores, as abelhas recolhem néctar e
pólen, ambos alimentos, e estes precisam ser adequados antes de
inseridos à colméia. Elas fazem regulação de umidade se necessário,
ingerindo água antes de coletá-los, tanto para consumo próprio, quanto
para mais facil transporte ao umidecer o pólen. Lagstroth resolveu
também este problema em grande parte, através da mobilidade que sua
arquitetura entrega. 


% \begin{drop 1}
O mel por sua vez, é produzido a partir da extração do néctar,
coletado na flora disponível, armazenado na colméia, regurgitado,
processado pelas enzimas presentes no corpo da abelha, e finalmente
desidratado (maturado) pelos próprios individuos. Quando a substância
alcança certo nível de umidade (em torno de 20\%), é considerada mel,
e é então é tampado para armazenamento. A célula, agora fechada e
preenchida de mel, recebe o adjetivo de operculada.

Observa-se aqui uma variação em características físicas da colméia
como um todo. O nectar in natura, ou "mel verde", possui cerca de 70\%
de sua massa composta por água. Já o mel maturado terá apenas metade
da massa do nectar. Esta é uma análise importante, pois aspectos da
função de variação de massa da colméia.

Outras flutuações em peso também ocorrem, devido à morte de individuos
da colônia, e abandono por diversos fatores. além da quantidade do
crescimento de novas crias, presença de polen, cera, temperatura e
umidade. No entanto, em média, a massa de uma colméia saudável cresce
até o limite da produção, determinado por características de cada
espécie e recursos disponíveis.

Portanto, parte do processo de criação de abelhas então consiste na
escolha de um bom ambiente, ou no controle direto e indireto desse,
levando em consideração flora e fauna disponível, provendo variedade
de nutrientes, além de nível de exposição ao sol, importante para
termoregulação da colméia, e disponibilidade de água, importante para
alimentação, termorregulação, e condicionamento do alimento. 

É então evidente que diversas medidas no processo produtivo do mel
pode ditar o ritmo com que a produção é realizada, variando da saúde
dos indivíduos, características climáticas, flora, aspectos
geográficos e espaciais.


% \end{drop 1}

Aquisição de dados e sensoriamento de colmeias

Sensores são dispositivos capazes de detectar mudanças no ambiente, e
gerar eventos sobre estas mudanças para um sistema. No meio apícola,
sensores tem ampla aplicação afim de detectar a maioria dos eventos
pertinentes para a colméia,

O processo aquisitório de dados pode incluir métodos invasivos (como a
inserção de sensores diretamente dentro da colméia, captura de
indivíduos, controle de entrada e saída, e afins), ou não invasivos
(monitoramento de temperatura externa, umidade do ar, peso, entre
outras). Métodos invasivos costumam ter resultados mais precisos, com
o contraponto de que podem interferir diretamente na acurácia dos
dados, em decorrência das alterações que o procedimento provoca no
ambiente, interferindo assim no comportamento natural dos indivíduos.
Métodos não invasivos costumam ser mais acurados, ao custo da
precisão, tendo em mente o espectro de desvios que o sensoriamento
sofre em decorrência de interferências do ambiente: vento, exposição
ao sol, esparcidade dos insetos, etc.

A ideia de sensoriar as colméias diretamente não é nova. Alguns
trabalhos demonstram que a ideia intriga pesquisadores à cerca de 20
anos.[?] No Brasil, apesar de recente, a ideia também tem se difundido
bastante, com o surgimento de pesquisas e soluções comerciais para
auxiliar a tomada de decisões.[?]

Dentre a enorme gama de aplicações, quatro análises se destacam:
sensoriamento de temperatura, umidade, peso e som.

A temperatura, bem como a umidade, como fator decisivo para o ritmo de
trabalho de todas as espécies, possui adoção quase que unanime entre
outros trabalhos.[?] A facilidade de implantação dos sensores,
processamento, e intepretação, permite extrair uma relação direta
entre a termorregulação da colmeia, ciclo de trabalho e eficiência
produtiva.[?]

Sua aplicação, no entanto, não tem uma discução direta em se tratando
da metodolgia. Simples variáveis, como por exemplo o posicionamento
dos sensores, podem alterar drasticamente as medidas. E os tipos de
interferências possíveis também pode ser um fator crucial.

Traba

O som, advindo do comportamento de enxame dessas criaturas, também
descreve um espectro de comportamento. É sabido que o som composto
pela colméia reflete tanto o estado atual, seja este fome, doença,
ameaça, quanto uma breve previsão de eventos, como enxameação, mudança
de ritmo produtivo, ou mesmo o compartilhamento de informações,
através do comportamento de "dança".[?]

Aplicações práticas demonstram a possibilidade de induzir certos
comportamentos, condicionando enxames inteiros à serem dóceis,
permitindo assim a coleta de mel sem a presença de riscos, sejam para
os coletores, quanto para a própria abelha.[?]

De maneira semelhante, os mesmos estudos no entanto, não analisam o
espectro das posiçções possíveis, bem como da diferença de som entre
as diversas espécies, ou o tipo de microfone. é conhecido que
diferentes espécies geram diferentes espectros de ruídos, de forma que
pode haver mais de uma forma de analizar determinada espécie através
do áudio.

E por fim, o peso evidencia a evolução da produção de alimentos e
indivíduos dentro da colmeia. Através da sua medida, o apicultor
consegue entender o ritmo de produção e consumo da colméia[12],
habilitando ações como alimentação artificial em épocas de escacez, ou
mesmo a melhor época para colheitas.

 
		Extração de audio pode automatizar a identificação de fatores
de stress. Para isso é necessário analizar os audios sistematicamente.
Existem modelos[Kulyukin, 2018] de redes neurais capazes classificar
sons de abelhas dentre outros sons. O projeto faz uso de sensores como
microfone, temperatura e câmera para tratar diferenciaçãode abelhas de
outros insetos, detecção e previsão de eventos na colmeia, detecção de
doenças, e associação de eventos à determinadas condições de produção

	Utilizando-se de caixas no padrão Langstroth, o estudo foi capaz
de e 

	Como aplicar?

	Quais os resultados?

Sensoriamentos internos

	Quais sensores?

	

	Como aplicar?

	Quais os resultados?

Pontos em comum




O que é I2C

O que é SPI





Apis melifera

A Apis mellifera, conhecida como abelha-europeia ou
abelha-africanizada no contexto brasileiro, é uma espécie exótica
amplamente distribuída e uma das mais estudadas do mundo. Pertencente
à ordem Hymenoptera e à família Apidae, essa abelha é reconhecida por
sua alta relevância ecológica e econômica, principalmente na
polinização de culturas agrícolas e na produção de mel, cera, própolis
e geleia real.[15]

Morfologicamente, a Apis mellifera distingue-se de muitas espécies
nativas por seu tamanho relativamente grande, variando entre 12 a 15
mm de comprimento, o que a torna significativamente maior do que a
maioria das abelhas sem ferrão (Meliponini), comuns no Brasil, que
geralmente medem entre 3 a 7 mm. Seu corpo é robusto, de coloração
predominantemente castanho-escura com listras abdominais amareladas ou
alaranjadas bem definidas, coberto por finos pelos que facilitam o
transporte de pólen. Suas asas são proporcionais ao corpo,
transparentes e com veias bem marcadas.

[Imagem apis mellifera]

Outro aspecto marcante é o ferrão funcional das operárias de Apis
mellifera, usado para defesa da colônia, ao contrário das espécies de
belhas sem ferrão, que evoluíram com mecanismos defensivos diferentes
e menos agressivos. Além disso, essa espécie apresenta um
comportamento defensivo mais intenso, principalmente em linhagens
africanizadas, o que facilita sua identificação comportamental em
campo.

A organização social da colônia é altamente complexa, composta por
três castas: uma rainha fértil, milhares de operárias estéreis e um
número reduzido de zangões. As colônias são perenes, podendo conter de
20 a 80 mil indivíduos, abrigadas geralmente em cavidades protegidas.
Esse nível de socialidade, aliado à sua capacidade de adaptação a
diferentes ambientes, contribuiu para sua expansão global e
consolidação como espécie dominante em apicultura.

Em contraste com muitas espécies nativas brasileiras, que são
frequentemente mais especializadas e sensíveis a distúrbios
ambientais, a Apis mellifera mostra-se generalista e resistente, o que
também levanta preocupações quanto ao seu impacto ecológico, sobretudo
em áreas de alta biodiversidade.


Tetragonisca angustula

A Tetragonisca angustula, popularmente conhecida como jataí, é uma
abelha social da tribo Meliponini, amplamente distribuída nas regiões
tropicais e subtropicais da América Latina, incluindo o Brasil.
Trata-se de uma das espécies de abelhas sem ferrão mais estudadas e
manejadas na meliponicultura, destacando-se por sua docilidade,
adaptabilidade e importância ecológica como polinizadora generalista .

Morfologicamente, a jataí é uma abelha de pequeno porte, medindo entre
4 a 5 mm de comprimento. Apresenta cabeça e tórax de coloração escura,
com abdômen mais claro e pernas pardacentas. Uma característica
distintiva é a ausência de ferrão funcional, comum a todas as abelhas
da tribo Meliponini, o que a torna incapaz de ferroar. Suas asas
possuem venação reduzida, e as operárias apresentam cerdas nas pernas
posteriores, adaptadas para o transporte de pólen .

[imagem jatai]

A entrada do ninho da T. angustula é notável, geralmente construída
com cera em forma de tubo, lembrando um dedo de luva, e pode
apresentar ramificações. Essa estrutura é utilizada na defesa da
colônia, sendo fechada quando há ameaça de invasores, como formigas ou
outras abelhas. As abelhas guardas, que representam cerca de 1 a 6% da
colônia, posicionam-se na entrada do ninho para monitorar e defender
contra intrusos .  

Ecologicamente, a jataí é uma espécie generalista, visitando uma ampla
variedade de plantas para coleta de néctar e pólen. Estudos indicam
que ela visita diversas famílias vegetais, como Asteraceae,
Euphorbiaceae, Moraceae, Fabaceae e Anacardiaceae, desempenhando papel
crucial na polinização de ecossistemas tropicais . 

A T. angustula é altamente adaptável, nidificando em cavidades
naturais, como ocos de árvores, e em ambientes urbanos, como paredes e
estruturas humanas. Sua ampla distribuição geográfica e capacidade de
adaptação a diferentes habitats contribuem para seu sucesso ecológico
e importância na conservação da biodiversidade . 

Fisicamente, Apis mellifera e Tetragonisca angustula apresentam
diferenças morfológicas nítidas e facilmente observáveis. A. mellifera
é uma abelha é maior. Seu tórax e abdômen são densamente recobertos
por pelos, o que contribui para a coleta de pólen, e suas asas são
proporcionalmente grandes, com venação bem marcada. Em contraste, T.
angustula é uma abelha de pequeno porte, com corpo mais delicado, e
menor quantidade de pelos. Suas asas são curtas e apresentam venação
reduzida, compatível com o seu tamanho corporal. Além disso, enquanto
A. mellifera possui ferrão funcional e estrutura física voltada para
defesa ativa, T. angustula não possui ferrão funcional, apresentando
comportamento defensivo passivo, compatível com sua anatomia menos
agressiva.

Apesar de sua ampla distribuição e relevância ecológica, T. angustula
ainda carece de uma base técnica consolidada na literatura científica,
especialmente no que diz respeito a parâmetros biométricos
padronizados, dados sobre produtividade, comportamento reprodutivo, e
manejo racional. Enquanto A. mellifera é alvo de centenas de estudos
experimentais e revisões sistemáticas, grande parte das informações
sobre a jataí encontra-se dispersa, muitas vezes restrita a fontes
regionais, relatos etnográficos ou experiências empíricas de
meliponicultores.

Assim, a comparação entre essas duas espécies não se limita ao campo
morfológico, mas serve também como alerta para a carência de dados
técnicos robustos sobre as abelhas nativas, o que limita seu uso
sustentável e sua conservação baseada em evidências científicas.
Investir na documentação sistemática de características como
morfologia, biometria, e comportamento de espécies como T. angustula é
um passo essencial para o fortalecimento da meliponicultura como
atividade científica e econômica.

Colmeias Langstroth e jatai

    A criação de ambas abelhas exigem a utilização de caixas
pré-montadas que permitem a modularização da coleta de mel. Com
modelos diferentes para cada uma das especies, o modelo ideal para as
abelhas européias seguem o padrão de Langstroth, e para as
abelhas-sem-ferrão, o modelo INPA, a saber

As caixas Langstroth são o modelo de colmeia racional mais amplamente
utilizado na apicultura moderna, especialmente para a criação de
abelhas do gênero Apis, como a Apis mellifera. Desenvolvidas em 1852
pelo pastor e apicultor norte-americano Lorenzo Lorraine Langstroth,
elas introduziram um princípio revolucionário: o “espaço de abelha”
(bee space), ou seja, um intervalo de aproximadamente 6 a 9 mm entre
os elementos internos da colmeia, suficiente para permitir a
movimentação das abelhas sem induzi-las a preencher o espaço com cera
ou própolis.

Esse conceito permitiu a criação de quadros móveis, que podem ser
retirados individualmente para inspeção, manejo ou extração de mel,
sem destruir a estrutura do ninho. Isso representou uma transformação
radical na apicultura, permitindo que os apicultores manejassem
colmeias com eficiência, mantivessem a saúde das abelhas sob controle
e extraíssem o mel sem matar a colônia, como ocorria em métodos mais
antigos.

Possui as seguintes partes

Ninho (corpo da colmeia):
Módulo inferior, onde a rainha deposita os ovos e se formam as crias. Contém geralmente de 8 a 10 quadros móveis, que as abelhas utilizam para criar favos de cria e armazenar pólen.

Melgueiras:
Módulos superiores semelhantes ao ninho, mas utilizados principalmente para armazenamento de mel. O apicultor pode empilhar duas ou mais melgueiras durante uma florada, aumentando a capacidade de produção sem perturbar a criação.

Quadros móveis:
São estruturas retangulares de madeira ou plástico, que podem conter cera alveolada pré-fabricada. As abelhas constroem seus favos sobre esses quadros, que podem ser facilmente removidos para colheita e revisão.

Fundo sanitário:
Base da colmeia, às vezes com tela para ventilação e controle de pragas como o ácaro Varroa.

Tampa interna e externa:
Ajudam a proteger a colmeia contra intempéries e controlar a temperatura interna.

Apesar de suas inúmeras vantagens, o modelo Langstroth é incompatível
com abelhas nativas sem ferrão, que têm comportamento e arquitetura de
ninho diferentes, como a formação de discos horizontais de cria. Isso
exige outros modelos, como a caixa modelo INPA, desenvolvida
especificamente para essas espécies.

[Imagem colmeia langstroth]

As caixas modelo INPA são colmeias racionais desenvolvidas no Brasil
para a criação de abelhas nativas sem ferrão, conhecidas como
meliponíneos. O nome “INPA” refere-se ao Instituto Nacional de
Pesquisas da Amazônia, onde esse modelo começou a ser utilizado e
disseminado, especialmente nas décadas de 1980 e 1990, como parte dos
esforços para promover a meliponicultura racional, ecológica e
adaptada às espécies nativas.

Diferentemente das caixas utilizadas na apicultura com Apis mellifera,
que adotam quadros verticais móveis, o modelo INPA respeita a
arquitetura natural dos ninhos das abelhas sem ferrão, que constroem
discos horizontais de cria empilhados, cercados por potes de mel e
pólen em estruturas orgânicas. As caixas INPA reproduzem essas
condições em um ambiente controlado e modular, permitindo o manejo
sustentável sem desorganizar o ninho.

As caixas INPA são compostas por módulos empilháveis, cada um com função distinta e dimensões adaptadas ao porte da espécie criada. Para a abelha Jataí, uma das menores espécies de meliponíneos, as medidas são reduzidas e delicadas. A estrutura básica inclui:

    Módulo de ninho: É o coração da colônia, onde são depositados os
ovos e construídos os discos de cria. Para Jataí, o módulo costuma
medir cerca de 15x15x9, considerando-se as paredes internas, com cerca
de 3cm de espessura. O espaço interno é fixo, sem divisórias móveis,
permitindo que as abelhas organizem o ninho de forma natural.

    Sobreninho (ou extensão do ninho): Módulo adicional posicionado
acima do ninho. Permite o crescimento da área de postura e
armazenamento, acomodando a expansão natural da colônia com o tempo.

    Melgueira: Espaço dedicado ao armazenamento de mel. Nela as
abelhas constroem pequenos potes de cera onde depositam o mel. A
extração é feita manualmente, com seringas ou espátulas, respeitando a
integridade da colônia.

    Fundo (com ou sem lixeira): Parte inferior da caixa. Pode ter um
espaço adicional para acumular resíduos (a chamada "lixeira"), que
facilita a limpeza periódica e o controle de pragas.

    Tampa e cobertura externa: Fecham a caixa superiormente, ajudando
a manter o microclima interno, protegendo contra luz, calor excessivo
e umidade.

[Fotografia colmeia jatai INPA]

Bufferização
