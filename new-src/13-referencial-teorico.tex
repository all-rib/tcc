\section{Abelhas, apicultura e apicultura de precisão}

As abelhas são insetos alados (clado Anthophila) responsáveis pelo processo da
polinização e produção de mel. As mais de 20 mil espécies, presentes em
todos os continentes com exceção da Antártida, se alimentan de pólen
e nectar, dando origem ao processo de polinização, impotante não
apenas para as plantas, mas para todo o ecossistema da produção de
alimentos, e principalmente, para a apicultura, atividade existente
desde o Egito Antigo, e de crescente importância no Brasil.

No Brasil, as espécies mais comuns compreendem a Apis mellifera
(Africanizada), Tetragonisca angustula (Jataí), e Eulaema nigrita
(Abelha das Orquídeas)[8], sendo aquela a mais numerosa, dado o
interesse do mercado brasileiro em sua alta produtividade para o
mercado apícola[10].

A domesticação do processo de produção de mel recebe o nome de
apicultura. É um processo milenar, praticado deste o Egito antigo, que
consiste em alocar as abelhas em uma região arbitária, proporcionar à
colmeia condições de crescimento (disponibilidade de alimento, água, e
condições climáticas adequadas), e extrair o mel periódicamente,
administrando também a criação de outras colméias, e competição entre
espécies.

É importante destacar que a apicultura geralmente se refere à produção
de mel num panorama geral, mas também pode significar a produção de
mel exclusivamente por abelhas do genero Apis (abelhas com ferrão), ou
tribo Apini, sendo sua maior representante a A. mellifera. Surge então
a meliponicultura, que trata da produção pela tribo Meliponini
(popularmente chamadas de abelhas sem ferrão). Para facilidade de
entendimento, tomar-se-à apicultura geral (APG) como sendo todo o processo
apícola, indistinto de espécies, e apicultura específica (APE), ou
simplesmente apicultuta como a produção de mel por abelhas do genero
Apis.

A APG teve início no Brasil na década de 1830, através da
importações de espécimes da Europa, a pedido do Pe. Antonio Pinto
Carneiro. A prática não era rentável, além de perigosa, e por algum
tempo foi considerada inviável, sendo retomada apenas em 1970, graças
a esforços de pesquisa apresentados no Congresso Brasileiro de
Apicultura, em Florianópolis.[5]

Cerca de 40 anos se passaram até que os métodos e tecnologias da
agricultura de precisão (AGP) fizessem luz à ao processo produtivo do
mel, dando origem então à apicultura de precisão (APP). A aplicação de
técnicas análogas de sensoriamento e controle comprovadamente otimizou
a produção de mel de diversas maneiras, entre impacto no ambiente, nas
espécies, quantidade de produtos (mel e própolis) e subprodutos
produzida (cera, medicamentos e cosméticos).

Pode-se definir a APP considerando-se alguns princípios da AGP: essa
trata da aquisição, processamento e análise de dados para permitir
otimizações no gerenciamento e na tomada de decisões do processo
agrícola. De maneira análoga, os mesmos estudos podem ser inseridos na
apicultura, se tornando então um processo de aquisição, processamento
e análise de dados afim de otimizar o gerenciamento e a tomada de
decisões no processo apícola.

\subsection{Tetragonisca angustula}

A Tetragonisca angustula, popularmente conhecida como jataí, é uma
abelha social da tribo Meliponini, amplamente distribuída nas regiões
tropicais e subtropicais da América Latina, incluindo o Brasil.
Trata-se de uma das espécies de abelhas sem ferrão mais estudadas e
manejadas na meliponicultura, destacando-se por sua docilidade,
adaptabilidade e importância ecológica como polinizadora generalista .

Morfologicamente, a jataí é uma abelha de pequeno porte, medindo entre
4 a 5 mm de comprimento. Apresenta cabeça e tórax de coloração escura,
com abdômen mais claro e pernas pardacentas. Uma característica
distintiva é a ausência de ferrão funcional, comum a todas as abelhas
da tribo Meliponini, o que a torna incapaz de ferroar. Suas asas
possuem venação reduzida, e as operárias apresentam cerdas nas pernas
posteriores, adaptadas para o transporte de pólen .

[imagem jatai]

Apesar de sua ampla distribuição e relevância ecológica, T. angustula
ainda carece de uma base técnica consolidada na literatura científica,
especialmente no que diz respeito a parâmetros biométricos
padronizados, dados sobre produtividade, comportamento reprodutivo, e
manejo racional.

A organização social da colônia é altamente complexa, composta por
várias castas, incluindo rainha, operárias e zangões. As colônias são perenes, podendo conter de
20 a 80 mil indivíduos, abrigadas geralmente em cavidades protegidas.

\subsection {Colmeias jatai}

A criação de ambas abelhas exigem a utilização de caixas pré-montadas
que permitem a modularização da coleta de mel. Com modelos diferentes
para cada uma das especies, o modelo ideal para as abelhas-sem-ferrão
é denominado modelo INPA.

As caixas modelo INPA foram desenvolvidas no Brasil para a criação de
abelhas nativas. O nome “INPA” refere-se ao Instituto Nacional de
Pesquisas da Amazônia, onde esse modelo começou a ser utilizado e
disseminado, especialmente nas décadas de 1980 e 1990, como parte dos
esforços para promover a meliponicultura racional, ecológica e
adaptada às espécies nativas.

Diferentemente das caixas utilizadas na apicultura com Apis mellifera,
que adotam quadros verticais móveis, o modelo INPA respeita a
arquitetura natural dos ninhos das abelhas sem ferrão, que constroem
discos horizontais de cria empilhados, cercados por potes de mel e
pólen em estruturas orgânicas. As caixas INPA reproduzem essas
condições em um ambiente controlado e modular, permitindo o manejo
sustentável sem desorganizar o ninho.

Esta é composta por módulos empilháveis, cada um com função distinta e
dimensões adaptadas ao porte da espécie criada. A estrutura básica
inclui:

Módulo de ninho: Onde são depositados os ovos e construídos os discos
de cria. Costuma medir cerca de 15x15x9, considerando-se as paredes
internas, com cerca de 3cm de espessura. O espaço interno é fixo, sem
divisórias móveis, permitindo que as abelhas organizem o ninho de
forma natural.

Sobreninho (ou extensão do ninho): Módulo adicional posicionado acima
do ninho. Permite o crescimento da área de postura e armazenamento,
acomodando a expansão natural da colônia com o tempo.

Melgueira: Espaço dedicado ao armazenamento de mel. Nela as abelhas
constroem pequenos potes de cera onde depositam o mel. A extração é
feita manualmente, com seringas ou espátulas, respeitando a
integridade da colônia.

Fundo (com ou sem lixeira): Parte inferior da caixa. Pode ter um
espaço adicional para acumular resíduos (a chamada "lixeira"), que
facilita a limpeza periódica e o controle de pragas.

Tampa e cobertura externa: Fecham a caixa superiormente, ajudando a
manter o microclima interno, protegendo contra luz, calor excessivo e
umidade.

[Fotografia colmeia jatai INPA]

\Section{Sesnores}

Sensores são dispositivos capazes de detectar mudanças no ambiente, e
gerar eventos sobre estas mudanças para um sistema. No meio apícola,
sensores tem ampla aplicação afim de detectar a maioria dos eventos
pertinentes para a colméia,

O processo aquisitório de dados pode incluir métodos invasivos (como a
inserção de sensores diretamente dentro da colméia, captura de
indivíduos, controle de entrada e saída, e afins), ou não invasivos
(monitoramento de temperatura externa, umidade do ar, peso, entre
outras). Métodos invasivos costumam ter resultados mais precisos, com
o contraponto de que podem interferir diretamente na acurácia dos
dados, em decorrência das alterações que o procedimento provoca no
ambiente, interferindo assim no comportamento natural dos indivíduos.
Métodos não invasivos costumam ser mais acurados, ao custo da
precisão, tendo em mente o espectro de desvios que o sensoriamento
sofre em decorrência de interferências do ambiente: vento, exposição
ao sol, esparcidade dos insetos, etc.

Dentre a enorme gama de aplicações, quatro análises se destacam:
sensoriamento de temperatura, umidade, peso e som.

Entre os muitos, cita-se especialmente BME280, DHT11, DS18B20, SHT30,
INMP441. Esses sensores cobrem a maior parte do escopo deste projeto,
a saber:

O sensor BME280, desenvolvido pela Bosch, é um dispositivo
ambiental de alta precisão projetado para medir pressão atmosférica,
temperatura e umidade relativa, integrando essas três
funcionalidades em um encapsulamento compacto do tipo LGA com
dimensões típicas de 2,5 x 2,5 x 0,93 mm. Baseado em tecnologia MEMS
(Micro-Electro-Mechanical Systems), o BME280 utiliza um princípio
piezoresistivo para detecção de pressão, um sensor capacitivo para
umidade e um sensor de banda de energia para temperatura,
proporcionando medições altamente estáveis e com baixo ruído. O
dispositivo suporta interfaces de comunicação digitais I2C e SPI,
permitindo fácil integração com microcontroladores e sistemas
embarcados. Seu intervalo operacional abrange temperaturas de -40 °C a
+85 °C, umidades de 0 a 100\% UR e pressões de 300 a 1100 hPa,
possibilitando o uso em aplicações meteorológicas, dispositivos IoT,
altímetros, estações ambientais e wearables. O consumo de energia é
otimizado por meio de modos de operação configuráveis (sleep, forced e
normal), o que o torna adequado para sistemas alimentados por bateria.
A compensação interna de temperatura e a calibração de fábrica
garantem a precisão das medições, que tipicamente alcançam ±1\% UR para
umidade absoluta, ±1 Pa para pressão absoluta e ±0,5 °C para temperatura.

[Imagem BME280]

O DHT11 é um sensor digital amplamente utilizado para medições básicas
de temperatura e umidade relativa do ar, composto por um elemento
resistivo de detecção de umidade e um termistor NTC encapsulados em
uma estrutura plástica com circuito de conversão analógico-digital
integrado. Seu princípio de funcionamento baseia-se na variação da
resistência elétrica do material sensível à umidade e na variação de
tensão do termistor em função da temperatura, sendo os sinais
convertidos e transmitidos por meio de uma interface digital de fio
único (single-wire) com protocolo proprietário, o que simplifica a
integração com microcontroladores de baixo custo. O dispositivo opera
em faixa de tensão típica de 3,3 V a 5,5 V, com corrente de repouso
inferior a 2,5 mA e tempo de amostragem mínimo de 1 segundo,
apresentando resolução de 1°C para temperatura e 1\% UR para umidade,
embora com precisão limitada (±2°C e ±5\% UR, respectivamente).
Contudo, é importante ressaltar que o DHT11 sofre com ampla incidência
de falsificações e variantes não oficiais, frequentemente vendidas sob
o mesmo nome mas com comportamento elétrico e características de
medição divergentes, o que dificulta a padronização e a confiabilidade
de resultados. Além disso, não há um fabricante claramente
identificado, e múltiplos datasheets circulam com especificações
ligeiramente diferentes, tornando essencial a verificação experimental
e a calibração manual para aplicações que exijam consistência ou
precisão. De maneira que este dispositivo, apesar de testado durante o
projeto, não será utilizado para resultados finais.

[Imagem DHT11]

O DS18B20 é um sensor digital de temperatura fabricado originalmente
pela Maxim Integrated, amplamente utilizado em aplicações de
monitoramento térmico devido à sua alta precisão, simplicidade de
interface e ampla faixa operacional. Baseado em tecnologia de
semicondutores, o DS18B20 mede temperatura através de um diodo
sensível integrado cujo coeficiente de variação de tensão é convertido
internamente em um valor digital, eliminando a necessidade de
conversão analógico-digital externa. O sensor utiliza o protocolo de
comunicação 1-Wire, permitindo a ligação de múltiplos dispositivos em
paralelo em um único barramento de dados, com cada unidade possuindo
um código serial único de 64 bits que facilita sua identificação e
endereçamento individual. Opera com alimentação entre 3,0 V e 5,5 V e
também suporta modo de alimentação parasita, no qual a energia é
derivada da linha de dados. Sua faixa de medição é de -55 °C a +125
°C, com precisão típica de ±0,5 °C no intervalo de -10 °C a +85 °C e
resolução configurável entre 9 e 12 bits, correspondendo a incrementos
mínimos de 0,0625 °C. O tempo de conversão depende da resolução
escolhida, variando de 93,75 ms a 750 ms. O encapsulamento mais comum
é o TO-92, embora existam versões em cápsulas estanques de aço
inoxidável para aplicações em líquidos e ambientes agressivos.  Graças
à sua robustez, baixo custo e facilidade de uso, o DS18B20 é
amplamente empregado em sistemas de controle ambiental, data loggers,
automação residencial e dispositivos IoT, sendo considerado uma das
soluções digitais mais confiáveis para medição de temperatura ponto a
ponto.

[Imagem DS18B20]

O SHT30 é um sensor digital de temperatura e umidade relativa do ar
desenvolvido pela Sensirion, projetado para oferecer alta precisão,
estabilidade a longo prazo e resposta rápida em aplicações ambientais
e industriais. Baseado na tecnologia CMOSens, o dispositivo integra em
um único chip o elemento sensor, o conversor analógico-digital e a
lógica de processamento e calibração, garantindo medições precisas e
linearizadas diretamente na saída digital. A comunicação com sistemas
embarcados ocorre por meio da interface I2C, suportando endereços
configuráveis e taxas de comunicação de até 1 MHz, o que facilita a
integração com microcontroladores modernos. O SHT30 opera em uma faixa
de tensão de 2,4 V a 5,5 V e mede temperaturas de -40 °C a +125 °C com
precisão típica de ±0,3 °C, e umidades relativas de 0 a 100\% UR com
precisão de ±2\% UR, apresentando tempo de resposta inferior a 8
segundos. O encapsulamento padrão é o DFN de 2,5 × 2,5 × 0,9 mm, com
membrana opcional de proteção contra condensação e contaminantes. O
sensor conta com calibração de fábrica armazenada em memória OTP
(One-Time Programmable), o que elimina a necessidade de ajustes
externos e assegura repetibilidade entre lotes. Devido à estabilidade
térmica, baixo consumo de energia e confiabilidade em ambientes com
variação de umidade, o SHT30 é amplamente utilizado em dispositivos
IoT, sistemas HVAC, automação industrial, estações meteorológicas e
equipamentos médicos, sendo reconhecido como uma das soluções mais
consistentes da sua categoria.

[Imagem SHT30]

O INMP441 é um microfone digital MEMS desenvolvido pela InvenSense,
projetado para captura de áudio de alta qualidade com baixo ruído e
saída digital direta. O dispositivo integra um elemento sensor de
pressão acústica baseado em tecnologia MEMS e um conversor
analógico-digital sigma-delta de 24 bits, fornecendo dados no formato
digital I²S (Inter-IC Sound), o que elimina a necessidade de circuitos
analógicos externos e simplifica a interface com microcontroladores e
processadores de sinal. Opera com tensão de alimentação entre 1,8 V e
3,3 V e consome tipicamente cerca de 1,4 mA em modo ativo, com
possibilidade de operação em modo de baixo consumo. O encapsulamento é
do tipo LGA, com dimensões de aproximadamente 4,72 × 3,76 × 1,0 mm, e
possui a porta acústica localizada na parte superior, facilitando a
montagem em aplicações portáteis e dispositivos com aberturas
frontais. O sensor apresenta resposta em frequência de 60 Hz a 15 kHz,
sensibilidade típica de -26 dBFS (referente a 94 dB SPL) e relação
sinal-ruído de cerca de 61 dBA, proporcionando captação limpa e
estável em aplicações de voz e detecção acústica. O INMP441 permite
configuração de canal esquerdo ou direito através do pino L/R,
possibilitando uso em sistemas estéreo ou arrays de microfones, e
utiliza sincronização padrão I²S com pinos BCLK, WS e SD. Graças à sua
precisão, baixo consumo e facilidade de integração digital, é
amplamente empregado em sistemas IoT, assistentes de voz, dispositivos
inteligentes, gravação de áudio embarcada e aplicações de
reconhecimento acústico.

[Imagem INMP441]

\Section{I2C}

O I²C (Inter-Integrated Circuit), também conhecido como I2C, é um
protocolo de comunicação serial síncrono utilizado em sistemas
embarcados para a troca de dados entre microcontroladores e
dispositivos periféricos. Desenvolvido originalmente pela Philips
Semiconductor na década de 1980, foi projetado para permitir a
comunicação de múltiplos dispositivos utilizando apenas duas linhas de
sinal, além da alimentação: uma linha de dados (SDA – Serial Data) e
outra de clock (SCL – Serial Clock).

[Diagrama de tempo do I2C]



% TODO: Rever abaixo o que jogar fora e o que manter



%A ideia de sensoriar as colméias diretamente não é nova. Alguns
%trabalhos demonstram que a ideia intriga pesquisadores à cerca de 20
%anos.[?] No Brasil, apesar de recente, a ideia também tem se difundido
%bastante, com o surgimento de pesquisas e soluções comerciais para
%auxiliar a tomada de decisões.[?]
%
%A temperatura, bem como a umidade, como fator decisivo para o ritmo de
%trabalho de todas as espécies, possui adoção quase que unanime entre
%outros trabalhos.[?] A facilidade de implantação dos sensores,
%processamento, e intepretação, permite extrair uma relação direta
%entre a termorregulação da colmeia, ciclo de trabalho e eficiência
%produtiva.[?]
%
%Sua aplicação, no entanto, não tem uma discução direta em se tratando
%da metodolgia. Simples variáveis, como por exemplo o posicionamento
%dos sensores, podem alterar drasticamente as medidas. E os tipos de
%interferências possíveis também pode ser um fator crucial.
%
%O som, advindo do comportamento de enxame dessas criaturas, também
%descreve um espectro de comportamento. É sabido que o som composto
%pela colméia reflete tanto o estado atual, seja este fome, doença,
%ameaça, quanto uma breve previsão de eventos, como enxameação, mudança
%de ritmo produtivo, ou mesmo o compartilhamento de informações,
%através do comportamento de "dança".[?]
%
%Aplicações práticas demonstram a possibilidade de induzir certos
%comportamentos, condicionando enxames inteiros à serem dóceis,
%permitindo assim a coleta de mel sem a presença de riscos, sejam para
%os coletores, quanto para a própria abelha.[?]
%
%De maneira semelhante, os mesmos estudos no entanto, não analisam o
%espectro das posiçções possíveis, bem como da diferença de som entre
%as diversas espécies, ou o tipo de microfone. é conhecido que
%diferentes espécies geram diferentes espectros de ruídos, de forma que
%pode haver mais de uma forma de analizar determinada espécie através
%do áudio.
%
%E por fim, o peso evidencia a evolução da produção de alimentos e
%indivíduos dentro da colmeia. Através da sua medida, o apicultor
%consegue entender o ritmo de produção e consumo da colméia[12],
%habilitando ações como alimentação artificial em épocas de escacez, ou
%mesmo a melhor época para colheitas.
%
%Extração de audio pode automatizar a identificação de fatores
%de stress. Para isso é necessário analizar os audios sistematicamente.
%Existem modelos[Kulyukin, 2018] de redes neurais capazes classificar
%sons de abelhas dentre outros sons. O projeto faz uso de sensores como
%microfone, temperatura e câmera para tratar diferenciaçãode abelhas de
%outros insetos, detecção e previsão de eventos na colmeia, detecção de
%doenças, e associação de eventos à determinadas condições de produção
%
%	Utilizando-se de caixas no padrão Langstroth, o estudo foi capaz
