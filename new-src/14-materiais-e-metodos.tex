
Considerando os objetivos do projeto, definiu-se as seguintes etapas:

Busca de metodologias de sensoriamento de abelhas A. mellifera,
considerando-se os dispositivos mais comumente utilizados (sensor de
temperatura, umidade, pressão, microfone, e luminosidade). A busca
será feita através de ScienceDirect e Scholar Google. Esta etapa
nomea-se Busca.

Estudo da natureza de ambas as abelhas, levantando considerações sobre
suas semelhanças e diferenças que podem interferir nos métodos
conforme disposições dos próprios artigos. Essa etapa nomeia-se
Estudos.

Aplicação dos métodos de monitoramento, e extração dos dados, em
colméias de T. angustula, observando principalmente a necessidade de
adaptação dos métodos e suas diferenças, para registro. Essa etapa
nomeia-se Aplicação e registro.

Observação dos resultados obtidos e comparação dos métodos,
levantando suas diferenças, e dificuldades do processo. Essa etapa
nomeia-se Análise-comparativa.

\subsection{Busca}

 Foram buscado os termos colméia, monitoramento, abelha, modelo,
sensoriamento, precisão, apicultura, observando-se suas respectivas traduções para o inglês
(honeycomb, monitoring, bee;honeybee, model, sensoring, precision,
apiculture,). Adiante
listam-se as considerações mais importantes encontradas.

[4] faz uma pesquisa extensiva nos mesmos tópicos, provendo uma
revisão de literatura abrangente sobre o assunto. Suas pesquisas citam
a importância da umidade para a saúde geral, considerando regulação
térmica, prevenção da proliferação de fungos e bactérias, e
preservação das propriedades do mel para consumo das crias. Além
disso, a temperatura tem papel importante na regulação das atividades
realizadas pelos indivíduos, seja para produção direta de mel, ou
procriação e crescimento de novos indivíduos.

	Segundo a mesma autora, o som também possui uma enorme
importância, pois é um dos canais de comunicação de ações em
comportamento de enxame, tanto quanto num instante pontual e não periódico,
como para a transmissão de eventos (alertas sobre invasores, fontes de
alimento, intempéries climáticas), ou para comportamento sazonal (como
aumento de atividade de evaporação de umidade, ou enxameação). Outras
variáveis como peso e composição atmosférica também interferem.

Algumas metodologias de sensoriamento envolvem Machine Learning,
Análise de dados e visão computacional, 

	[trindade,2018] utiliza um sensor de temperatura e umidade, alem
de celulas para medir peso. Para transmissao dos dados, foi utilizado
um dispositivo Xbee, mas nao descreve detalhes sobre qualidade de
sinal, ou mesmo sobre o tipo de servidor utilizado, apontando apenas
que os dados foram processados utilizando-se microsoft excel.

	Estendendo a analise da obra de maneira hipotetica, pode se
estipular que parte da decisao de utilizar um dispositivo Xbee foi
devido a sua facilidade de conexao com a rede, por ser facilmente
interfaceavel com outros sistemas operacionais. Torna-se entao
evidente a necessidade de um sinal de telefonia movel, ou sinal de
internet para a transmissao dos dados para nuvem.

	Alem disso, trabalhos como [ferrari,2008, entre outros] possuem
uma taxa de captura muito maior de dados, estendendo não apenas
variáveis como temperatura e umidade, mas também microfone, e imagem,
o que cria um enorme fluxo de dados para serem processados,
inviabilizando o processamento local.

Análise de temperatura

	Trabalhos 




Descrição da caixa de abelha





	Os sistemas idealmente precisam sensoriar ao menos um ponto
dentro, e um ponto fora da colmeia, permitindo assim observar a
diferen;a de temperatura.

	Os sensores utilizados foram os sht30

	Escolhidos pela disponibilidade e precisao, alem de possuírem um
sensor de umidade embarcado. seu tambanho também facilita o
posicitionamento em qualquer local da colmeia

I2C

\subsection{Estudos}

\subsubsection{Temperatura}



\subsection{Aplicação e registro}
\subsection{Análise Comparativa}




