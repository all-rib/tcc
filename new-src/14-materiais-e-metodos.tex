
Considerando os objetivos do projeto, definiu-se as seguintes etapas:

Busca de metodologias de sensoriamento de abelhas A. mellifera,
considerando-se os dispositivos mais comumente utilizados (sensor de
temperatura, umidade, pressão, microfone, e luminosidade). A busca
será feita através de ScienceDirect e Scholar Google. Esta etapa
nomea-se Busca.

Estudo da natureza de ambas as abelhas, levantando considerações sobre
suas semelhanças e diferenças que podem interferir nos métodos
conforme disposições dos próprios artigos. Essa etapa nomeia-se
Estudos.

Aplicação dos métodos de monitoramento, e extração dos dados, em
colméias de T. angustula, observando principalmente a necessidade de
adaptação dos métodos e suas diferenças, para registro. Essa etapa
nomeia-se Aplicação e registro.

Observação dos resultados obtidos e comparação dos métodos,
levantando suas diferenças, e dificuldades do processo. Essa etapa
nomeia-se Análise-comparativa.

\subsection{Busca}

 Foram buscado os termos colméia, monitoramento, abelha, modelo,
sensoriamento, precisão, apicultura, observando-se suas respectivas traduções para o inglês
(honeycomb, monitoring, bee;honeybee, model, sensoring, precision,
apiculture,). Adiante
listam-se as considerações mais importantes encontradas.

[4] faz uma pesquisa extensiva nos mesmos tópicos, provendo uma
revisão de literatura abrangente sobre o assunto. Suas pesquisas citam
a importância da umidade para a saúde geral, considerando regulação
térmica, prevenção da proliferação de fungos e bactérias, e
preservação das propriedades do mel para consumo das crias. Além
disso, a temperatura tem papel importante na regulação das atividades
realizadas pelos indivíduos, seja para produção direta de mel, ou
procriação e crescimento de novos indivíduos.

	Segundo a mesma autora, o som também possui uma enorme
importância, pois é um dos canais de comunicação de ações em
comportamento de enxame, tanto quanto num instante pontual e não periódico,
como para a transmissão de eventos (alertas sobre invasores, fontes de
alimento, intempéries climáticas), ou para comportamento sazonal (como
aumento de atividade de evaporação de umidade, ou enxameação). Outras
variáveis como peso e composição atmosférica também interferem.

Algumas metodologias de sensoriamento envolvem Machine Learning,
Análise de dados e visão computacional, 

	[trindade,2018] utiliza um sensor de temperatura e umidade, alem
de celulas para medir peso. Para transmissao dos dados, foi utilizado
um dispositivo Xbee, mas nao descreve detalhes sobre qualidade de
sinal, ou mesmo sobre o tipo de servidor utilizado, apontando apenas
que os dados foram processados utilizando-se microsoft excel.

	Estendendo a analise da obra de maneira hipotetica, pode se
estipular que parte da decisao de utilizar um dispositivo Xbee foi
devido a sua facilidade de conexao com a rede, por ser facilmente
interfaceavel com outros sistemas operacionais. Torna-se entao
evidente a necessidade de um sinal de telefonia movel, ou sinal de
internet para a transmissao dos dados para nuvem.

	Alem disso, trabalhos como [ferrari,2008, entre outros] possuem
uma taxa de captura muito maior de dados, estendendo não apenas
variáveis como temperatura e umidade, mas também microfone, e imagem,
o que cria um enorme fluxo de dados para serem processados,
inviabilizando o processamento local.

Análise de temperatura

	Trabalhos 





	Os sistemas idealmente precisam sensoriar ao menos um ponto
dentro, e um ponto fora da colmeia, permitindo assim observar a
diferen;a de temperatura. Isso se deve ao fato de que a variação de
temperatura do ambiente como um todo provoca alterações de
comportamento. Para manutenção da colméia, as abelhas fazem movimentos
para controle de umidade e temperatura, reduzindo assim efeitos desses
agentes sobre o ecossistema. Alterações no comportamento podem ser
observadas através da observação destes fenômenos, além de permitir a
identificação de fenômenos que provoquem estes comportamentos.

	Os sensores utilizados foram os sht30 para os ambientes externos.

O SHT30, desenvolvido pela Sensirion, é um sensor digital de
temperatura e umidade relativa altamente integrado, ideal para
aplicações em sistemas embarcados. Ele utiliza comunicação via I²C com
suporte a verificação de integridade CRC, o que o torna confiável
mesmo em ambientes com interferência elétrica. Possui calibração de
fábrica, o que elimina a necessidade de ajustes manuais, e oferece
alta precisão, com erro típico de ±2\% para umidade relativa e ±0,3 °C
para temperatura. Com baixo consumo de energia e encapsulamento
compacto (DFN-6), o SHT30 é amplamente utilizado em aplicações
industriais, automação residencial, dispositivos IoT e monitoramento
ambiental. Sua combinação de precisão, estabilidade e facilidade de
integração o torna uma escolha robusta e eficiente para projetos que
exigem sensoriamento ambiental confiável.

	Escolhidos pela disponibilidade e precisao, alem de possuírem um
sensor de umidade embarcado. seu tambanho também facilita o
posicitionamento em qualquer local da colmeia. O sensor possui
interface I2C, com velocidades de ate 1MHz, com acuracia de 1.5\% de
umidade relativa, e 0.1oc. Seu baixo consumo (1.5mA durante medicao, e
0.5uA em idle-mode) tambem e um fator
importante para a aplicacao, considerando que o dispositivo deve rodar
sem supervisao durante um grande periodo, utilizando baterias. Por
fim, o dispositivo nao necessita de calibracao.

No entanto, sua breakout board possui um circuito pequeno que pode
atrabalhar a insercao direta do sensor dentro da colmeia. Em
decorrencia de eventuais falhas, um foi posto do lado de fora, outro
do lado de dentro.

%TODO: revisar isso segundo datasheet()

	Para comparacao. e possibilidade de contato direto com o mel,
escolheu-se o DS18b20 por sua versatilivdade de encapsulamentos.

O DS18B20 é um sensor digital de temperatura fabricado pela Maxim Integrated (atualmente parte da Texas Instruments), amplamente utilizado em sistemas embarcados e aplicações industriais. Seu destaque está na comunicação via 1-Wire, que permite a transmissão de dados e alimentação por um único fio, simplificando o cabeamento em sistemas distribuídos. O sensor oferece uma faixa de medição de -55 °C a +125 °C, com precisão típica de ±0,5 °C na faixa entre -10 °C e +85 °C, e resolução configurável de 9 a 12 bits. Além disso, cada unidade possui um código único de 64 bits gravado de fábrica, permitindo a identificação individual de múltiplos sensores conectados em paralelo no mesmo barramento. Seu encapsulamento compacto (TO-92 ou versões estanques em aço inox) e sua robustez tornam o DS18B20 ideal para uso em ambientes externos, automação residencial, agricultura de precisão e sistemas de monitoramento térmico distribuído.

	Para microfone, foi usado o INMP441, devido a disponibilidade,
familiaridade, e aplicacoes ja implementadas com esse mesmo
dispositivo. 

O INMP441 é um microfone digital do tipo MEMS
(Micro-Electro-Mechanical Systems), projetado para capturar sinais
acústicos com alta fidelidade e baixo ruído. Ele integra internamente
um conversor ADC de 24 bits e se comunica por meio da interface
digital I²S, permitindo a transmissão direta de áudio para
microcontroladores ou processadores de sinal sem a necessidade de
circuitos analógicos externos. O INMP441 apresenta uma faixa dinâmica
ampla e resposta em frequência plana, sendo capaz de captar sons entre
60 Hz e 15 kHz com boa sensibilidade. Sua arquitetura digital elimina
interferências comuns a microfones analógicos, tornando-o ideal para
aplicações como reconhecimento de voz, gravação de áudio embarcada,
sistemas de monitoramento acústico e dispositivos IoT com entrada
sonora. Com encapsulamento compacto e fácil integração em placas de
circuito impresso, o INMP441 oferece uma solução prática e robusta
para projetos embarcados que exigem captura de áudio de forma precisa
e eficiente.

	Para processamento e armazenamento das informacoes, foi utilizado
uma placa LORA32 da LILYGO, devida ao seu hardware integrado para uso
de cartoes SD. isso permite a gravacao dos dados durante um grande
periodo, se considerar a disponibilidade de armazenamento no que tange
cartoes de memoria do tipo microSD

I2C

\subsection{Estudos}

\subsubsection{Temperatura}



\subsection{Aplicação e registro}
\subsection{Análise Comparativa}




