\begin{abstractutfpr}
This work aims to study, apply, and adapt hive monitoring methods for Tetragonisca angustula (jataí) bees through the development of an environmental sensing architecture. The research involved implementing an architecture based on an ESP32 LoRa32 microcontroller, integrated with temperature, humidity, pressure, and sound sensors, using I²C communication and local storage. Bench and field tests were conducted to analyze the interaction between electronic components and the organic environment of the hive. The main difficulties observed included interference on the communication bus, intermittent sensor failures, and memory and power limitations. As a solution, a combination of hardware and software adaptations was proposed, including encapsulated sensor designs and the development of the Hermes and Sparse-Buffer libraries, focused on data integrity, real-time concurrency control, and reduced memory usage. The results showed that problems of this kind of integration are non-trivial, highlighting the need for further research on sensor isolation and durability under organic conditions. This work contributes to the advancement of national precision beekeeping, providing a practical foundation for future monitoring platforms for native bees.

\end{abstractutfpr}
