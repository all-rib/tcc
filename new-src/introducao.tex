
A apicultura é parte essencial da agroindustria brasileira. Responsáveis por 73\%
da flora nacional, as abelhas são imprescindíveis para um equilíbrio
natural, produção de alimentos, e preservação do meio ambiente[1]. Em 2022, a Associação Brasileira de Estudo das Abelhas (A.B.E.L.H.A)
publicou o Atlas da Apicultura no Brasil[2]. Esta ferramenta reúne dados de
instituições como a Organização das Nações Unidas da Alimentação e Agricultura (FAO)
e o Instituto Brasileiro de Geografia e Estatística (IBGE). FAO aponta o Brasil como
22º produtor mundial, tendo a produção aumentado em cerca de 12\% entre 2021 e
2022.[3] Um crescimento significativo, mesmo durante a pandemia. 

Paralelo a isso, existe um processo corrente que consiste na informatização da
produção de mel, utilizando tecnologias para otimizar a extração do mel, em quantidade e qualidade, simultaneamente reduzindo o
impacto causado pela atividade, que atualmente é realizada manualmente. A chamada Apicultura de
Precisão utiliza sensores, modelos de comportamento e microgerenciamento[4], uma
ideia já aplicada na agricultura de precisão. Porém é um tema novo no Brasil, quando
comparado a outros países pelo mundo[4].

Em pesquisas no corpo nacional de publicações[4], os maiores esforços ainda
desconsidera em grande parte a utilização das técnicas de controle e monitoramento de
sistemas para aplicação direta na produção de mel, seja pela falta de dados no que
tange às espécies nacionais de abelhas, seja na falta de esforços especializados na
área.

Assim sendo, este o presente trabalho se propõe elaborar um sistema de monitoramento
para colméias, expondo problemas e desenvolvendo técnicas adequadas para interfacear
o uso da tecnologia para a extração de um recurso tão rico quanto o mel.

O sistema, simples, minimamente invasivo e resiliente, precisa ter facilidade de
instalação, manutenção, e telemetria, para fazer berço aos futuros projetos
relacionados.




