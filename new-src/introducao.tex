
A apicultura é um tópico crescente na agroindustria brasileira. Responsáveis por 73\%
da flora brasileira, estes pequenos insetos são imprescindíveis para um equilíbrio
natural[1]. Em 2022, a Associação Brasileira de Estudo das Abelhas (A.B.E.L.H.A)
publicou o Atlas da Apicultura no Brasil. Esta ferramenta é responsável por agregar
dados da Organização das Nações Unidas da Alimentação e Agricultura (FAO), Instituto
Brasileiro de Geografia e Estatística (IBGE), entre outros. FAO aponta o Brasil como
22º produtor mundial, tendo a produção aumentado em cerca de 12\% entre 2021 e
2022.[2] Um crescimento significativo, mesmo durante a pandemia.

em paralelo a isso, existe um processo corrente da informatização da cultura do mel,
dando luz à área de apicultura de precisão, que consiste em utilizar tecnologias para
otimizar a extração do mel, em quantidade e qualidade, simultaneamente reduzindo o
impacto da extração, quando comparado à extração manual. Tais processos utilizam
sensores, modelos de comportamento e microgerenciamento[4], uma ideia já aplicada na
agricultura de precisão. Porém é um tema novo quando comparado a outros países pelo
mundo[4].

Por fim, sem uma abordagem clara de uma plataforma única para fazê-la, propõe-se aqui
o desenvolvimento de um sistema simples, e pouco invasivo, resiliente, e de fácil
instalação, manutenção, e monitoramento para fazer berço aos futuros desenvolvimentos
da área.
