

Abelhas e apicultura

As abelhas são insetos alados, responsáveis pelo processo da
polinização e produção de mel. As diversas espécies, presentes em
todos os continentes, com exceção da Antártida, se alimentan de pólen
e nectar, dando origem ao processo de polinização, impotante não
apenas para as plantas, mas para todo o ecossistema da produção de
alimentos, e principalmente, para a apicultura, atividade existente
desde o Egito Antigo, e de crescente importância no Brasil.

No Brasil, as espécies mais comuns compreendem a Apis mellifera
(Africanizada), Tetragonisca angustula (Jataí), e Eulaema nigrita
(Abelha das Orquídeas)[8], sendo aquela a mais numerosa, dado o
interesse do mercado brasileiro em sua alta produtividade para o
mercado apícola[10].

É importante diferenciar duas facetas do mercado apícola: a
apicultura, e a meliponicultura. A apicultura trata da produção de mel
exclusivamente por abelhas do genero Apis (abelhas com ferrão), ou
tribo Apini, sendo sua maior representante a A.  mellifera. Já a
meliponicultura trata da produção pela tribo Meliponini (popularmente
chamadas de abelhas sem ferrão). 

A apicultura teve início no Brasil na década de 1830, através da
importações de espécimes da Europa, a pedido do Pe. Antonio Pinto
Carneiro. A prática não era rentável, além de perigosa, e por algum
tempo foi considerada inviável, sendo retomada apenas em 1970, graças
a esforços de pesquisa apresentados no Congresso Brasileiro de
Apicultura, em Florianópolis.[5]

Cerca de 40 anos se passaram até que os métodos e tecnologias da
agricultura de precisão (AGP) fizessem luz à ao processo produtivo do
mel, dando origem então à apicultura de precisão (APP). A aplicação de
técnicas análogas de sensoriamento e controle comprovadamente otimizou
a produção de mel de diversas maneiras, entre impacto no ambiente, nas
espécies, quantidade de produtos (mel e própolis) e subprodutos
produzida (cera, medicamentos e cosméticos).

Pode-se definir a APP considerando-se alguns princípios da AGP: essa
trata da aquisição, processamento e análise de dados para permitir
otimizações no gerenciamento e na tomada de decisões do processo
agrícola. De maneira análoga, os mesmos estudos podem ser inseridos na
apicultura.


Técnicas rudimentares para otimização produtiva

O produção de mel para as abelhas é quase um todo da sua
sobrevivência, tendo em vista sua função alimentar para os indivíduos,
e suas crias.  Idealmente, elas não cessam a produção de mel, e
enquanto houver disponibilidade de recursos, e espaço para novas
crias, a colméia seguirá crescendo.

Mas na realidade, aspectos como estação do ano, clima, doenças, e
competição com outras espécies, acabam por determinar um passo de
cresimento com limites bem definidos[11], e portanto, o controle sobre
esses aspectos ditam o ápice produtivo da colmeia.

A observação dessas variáveis levou o homem à desenvolver tecnicas e
tenologias para o controle. Históricamente, isso se deu principalmente
a partir do século 19. Fatores como espaço, exposição ao clima, e
disponibilidade de alimento são elementos facilmente controláveis pelo
produtor, e seu estudo é mister para o processo produtivo.

Como um passo inicial, observa-se a padronização da colmeia. Concebida
por Lorenzo L. Langstroth, em 1851, os padrões de medida da caixa
otimizaram o cresimento e manutenção das colméias, tendo em vista a
densidade de produção (relação entre espaço livre, e espaço produtivo
necessário) natural que as abelhas exigem[6]. Esse padrão, geralmente
implementado para crias de espécie \it{Apis mellifera} (abelha-europa,
ou abelha-comum) se adequa à uma separação natural de funções que as
abelhas realizam: a parte inferior da colméia comporta postura e
crias, enquanto as superiores servem para alimentação e reservas.
Então o crescimento da colméia pode ser mais facilmente controlado e
monitorado pelo apicultor.

A arquitetura de Langstroth também altera a forma natural com que as
abelhas constroem os favos. Naturalmente, eles são curvados, de
maneira que a extração do mel é extremamente danosa para a estrutura
da colméia em sí.  Já dentro dos apiários, de maneira estruturada, as
caixas possuem quadros sobre os quais as abelhas podem construir
paredes retas e de dimensões bem definidas, o que facilita a
colheita[6].

Outra variável é a exposição solar. Muita expoxisão ao sol significa
baixa umidade, e nível maior de calor, dificultando a sobrevivência no
verão. Ao passo que pouca significará dificuldade de orientação, e
temperaturas mais baixas, dificultando a sobrevivência no inverno.
Durante as épocas de abertura das flores, as abelhas recolhem néctar e
pólen, ambos alimentos, e estes precisam ser adequados antes de
inseridos à colméia. Elas fazem regulação de umidade se necessário,
ingerindo água antes de coletá-los, tanto para consumo próprio, quanto
para mais facil transporte ao umidecer o pólen. Lagstroth resolveu
também este problema em grande parte, através da mobilidade que sua
arquitetura entrega. 

O mel por sua vez, é produzido a partir da extração do néctar, coletado na flora
disponível, armazenado na colméia, regurgitado, processado pelas
enzimas presentes no corpo da abelha, e finalmente desidratado
(maturado) pelos próprios individuos. Quando a substância alcança
certo nível de umidade (em torno de 20\%), é considerada mel, e é
então é tampado para armazenamento. A célula, agora fechada e
preenchida de mel, recebe o adjetivo de operculada.

Observa-se aqui uma variação em características físicas da colméia
como um todo. O nectar in natura, ou "mel verde", possui cerca de 70\%
de sua massa composta por água. Já o mel maturado terá apenas metade
da massa do nectar. Esta é uma análise importante, pois aspectos da
função de variação de massa da colméia.

Outras flutuações em peso também ocorrem, devido à morte de individuos
da colônia, e abandono por diversos fatores. além da quantidade do
crescimento de novas crias, presença de polen, cera, temperatura e
umidade. No entanto, em média, a massa de uma colméia saudável cresce
até o limite da produção, determinado por características de cada
espécie e recursos disponíveis.

Portanto, parte do processo de criação de abelhas então consiste na
escolha de um bom ambiente, ou no controle direto e indireto desse,
levando em consideração flora e fauna disponível, provendo variedade
de nutrientes, além de nível de exposição ao sol, importante para
termoregulação da colméia, e disponibilidade de água, importante para
alimentação, termorregulação, e condicionamento do alimento. 

É então evidente que diversas medidas no processo produtivo do mel
pode ditar o ritmo com que a produção é realizada, variando da saúde
dos indivíduos, características climáticas, flora, aspectos
geográficos e espaciais.


Aquisição de dados e sensoriamento de colmeias

O processo aquisitório de dados pode incluir métodos invasivos (como a
inserção de sensores diretamente dentro da colméia, captura de
indivíduos, controle de entrada e saída, e afins), ou não invasivos
(monitoramento de temperatura externa, umidade do ar, peso, entre
outras). Métodos invasivos costumam ter resultados mais precisos, com
o contraponto de que podem interferir diretamente na acurácia dos
dados, em decorrência das alterações que o procedimento provoca no
ambiente, interferindo assim no comportamento natural dos indivíduos.
Métodos não invasivos costumam ser mais acurados, ao custo da
precisão, tendo em mente o espectro de desvios que o sensoriamento
sofre em decorrência de interferências do ambiente: vento, exposição
ao sol, esparcidade dos insetos, etc.

A ideia de sensoriar as colméias diretamente não é nova. Alguns
trabalhos demonstram que a ideia intriga pesquisadores à cerca de 20
anos.[?] No Brasil, apesar de recente, a ideia também tem se difundido
bastante, com o surgimento de pesquisas e soluções comerciais para
 auxiliar a tomada de decisões.[?]

Dentre a enorme gama de aplicações, quatro análises se destacam:
sensoriamento de temperatura, umidade, peso e som.

A temperatura, bem como a umidade, como fator decisivo para o ritmo de trabalho de todas
as espécies, possui adoção quase que unanime entre outros trabalhos.[?] A facilidade de
implantação dos sensores, processamento, e intepretação, permite
extrair uma relação direta entre a termorregulação da colmeia, ciclo
de trabalho e eficiência produtiva.[?]

Sua aplicação, no entando, não tem uma discução direta em se tratando
da metodolgia: onde posicionar os sensores? Dentro, ou fora? o mel
interfere? pode, ser colocados em contato com a cera de abelha? Os
sensores interferem no comportamento dos indivíduos? Um estudo
especializado é necessário

O som, advindo do comportamento de enxame dessas criaturas, também
descreve um espectro de comportamento. É sabido que o som composto
pela colméia reflete tanto o estado atual, seja este fome, doença,
ameaça, quanto uma breve previsão de eventos, como enxameação, mudança
de ritmo produtivo, ou mesmo o compartilhamento de informações,
através do comportamento de "dança".[?]

Aplicações práticas demonstram a possibilidade de induzir certos
comportamentos, condicionando enxames inteiros à serem dóceis,
permitindo assim a coleta de mel sem a presença de riscos, sejam para
os coletores, quanto para a própria abelha.[?]

De maneira semelhante, os mesmos estudos no entanto, não analisam o
espectro das posiçções possíveis, bem como da diferença de som entre
as diversas espécies, ou o tipo de microfone. é conhecido que
diferentes espécies geram diferentes espectros de ruídos, de forma que
pode haver mais de uma forma de analizar determinada espécie através
do áudio.

E por fim, o peso evidencia a evolução da produção de alimentos e
indivíduos dentro da colmeia. Através da sua medida, o apicultor
consegue entender o ritmo de produção e consumo da colméia[12],
habilitando ações como alimentação artificial em épocas de escacez, ou
mesmo a melhor época para colheitas.


Metodologias de sensoriamento

Limitando-se ao escopo deste trabalho, é importante estudar metologias
da implementação de sensores, sejam eles de áudio, peso, temperatura, 


Sensoriamentos externos

	Quais sensores?

		Extração de audio pode automatizar a identificação de fatores
de stress. Para isso é necessário analizar os audios sistematicamente.
Existem modelos[Kulyukin, 2018] de redes neurais capazes classificar
sons de abelhas dentre outros sons. O projeto faz uso de sensores como
microfone, temperatura e câmera para tratar diferenciaçãode abelhas de
outros insetos, detecção e previsão de eventos na colmeia, detecção de
doenças, e associação de eventos à determinadas condições de produção

	O sujeito utilizou uma caixa no padrão Langstroth, ideal para
criação de abelhas por causa da modularidade das funções, e 

	Como aplicar?

	Quais os resultados?

Sensoriamentos internos

	Quais sensores?

	

	Como aplicar?

	Quais os resultados?

Pontos em comum


Problema

Objetivo

Metodologia

Resultados

Conclusão











O que é I2C

O que é SPI

Bufferização


[1] https://www.grap.udl.cat/en/presentacio/que-fem/definicions-agricultura-de-precisio/
23/01/2025, 00:38


