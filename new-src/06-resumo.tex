\begin{resumoutfpr}

Este trabalho pretende estudar, aplicar e adaptar métodos de
monitoramento de colmeias de abelhas em \textit{Tetragonisca
angustula} (jataí), através do desenvolvimento de uma arquitetura de
sensoriamento ambiental. A pesquisa envolveu a implementação de uma
arquitetura baseada em microcontrolador ESP32 LoRa32, integrado a
sensores de temperatura, umidade, pressão e som, utilizando
comunicação I²C e armazenamento local. Foram realizados testes em
bancada e em campo, analisando a interação entre componentes
eletrônicos e o ambiente orgânico da colmeia. As principais
dificuldades observadas incluíram interferências no barramento de
comunicação, falhas intermitentes em sensores e limitações de memória
e energia. Como solução, foi proposto uma amálgama entre adaptações de
hardware e software, incluindo encapsulamento adaptado, e o
desenvolvimento das bibliotecas \textbf{Hermes} e
\textbf{Sparse-Buffer}, voltadas à integridade de dados e controle de
concorrência em tempo real, e redução de carga no uso de memória. Como
resultado, foi evidenciado que os problemas com esse tipo de
integração não é trivial, ressaltando a necessidade de novas pesquisas
sobre isolamento de sensores e durabilidade em condições orgânicas. O
trabalho contribui para a consolidação da apicultura de precisão
nacional, oferecendo uma base prática para futuras plataformas de
monitoramento de abelhas nativas.

\end{resumoutfpr}
