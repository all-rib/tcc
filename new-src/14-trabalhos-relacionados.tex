Para encontrar outros trabalhos pertinentes, foram buscado os termos
colméia, monitoramento, abelha, modelo, sensoriamento, precisão,
apicultura, observando-se suas respectivas traduções para o inglês
(honeycomb, monitoring, bee;honeybee, model, sensoring, precision,
apiculture,). Adiante listam-se as considerações mais importantes
encontradas.

[4] faz uma pesquisa extensiva nos mesmos tópicos, provendo uma
revisão de literatura abrangente sobre o assunto. Suas pesquisas citam
a importância da umidade para a saúde geral da colméia, e manutenção
das crias, regulação térmica, prevenção da proliferação de fungos e
bactérias, e preservação das propriedades do mel para consumo das
crias. Além disso, a temperatura possui funções semelhantes, de
maneira que a maioria das operarias realizam atividades para
manutenção constante dessas variáveis.

Segundo a mesma autora, o som também possui uma enorme importância,
pois é um dos canais de comunicação para ações do enxame como um todo,
responsável pela transmissão de eventos (alertas sobre invasores,
fontes de alimento, intempéries climáticas) e comportamento sazonal
(como aumento de atividade de evaporação de umidade, ou enxameação).
Outras variáveis como peso e composição atmosférica também interferem
nas características de ruído da colméia.

Algumas metodologias de sensoriamento envolvem Machine Learning,
Análise de dados e visão computacional, a exemplo, [trindade,2018]
utiliza um sensor de temperatura e umidade, alem de celulas para medir
peso. Para transmissao dos dados, foi utilizado um dispositivo Xbee,
mas nao descreve detalhes sobre qualidade de sinal, ou mesmo sobre o
tipo de servidor utilizado, apontando apenas que os dados foram
processados utilizando-se microsoft excel.

Estendendo a analise da obra de maneira hipotetica, pode se estipular
que parte da decisao de utilizar um dispositivo Xbee foi devido a sua
facilidade de conexao com a rede, por ser facilmente interfaceavel com
outros sistemas operacionais. Torna-se então evidente a necessidade de
um sinal de telefonia movel, ou sinal de internet para a transmissao
dos dados para nuvem.

Alem disso, trabalhos como [ferrari,2008] possuem uma taxa de captura
muito maior de dados, estendendo não apenas variáveis como temperatura
e umidade, mas também microfone, e imagem, o que cria um enorme fluxo
de dados para serem processados, inviabilizando o processamento local.
Para endereçar esse problema, armmazenimento externo será utilizado.

Portanto, percebe-se que um caminho para a criação de um perfil pode
envolver temperatura, som, umidade, pressão, características que podem
refletir eventos como a presença ou a falta de alimento, a presença de
predadores, níveis de saúde, e eventos específicos como a enxameação.
No entanto, a analise dessas obras demonstra a limitação em apresentar
resultados conceituais ou de laboratório, sem descrever os desafios
enfrentados na implementação prática. Essa ausência de informações
técnicas dificulta a reprodutibilidade e o avanço incremental das
soluções propostas.

Outro ponto recorrente é a falta de discussão sobre os problemas reais
de campo — como interferências elétricas, falhas na comunicação e
instabilidades na alimentação. Nenhuma das obras abordam estratégias
para mitigar esses fatores, o que indica uma lacuna significativa
entre a teoria e a aplicação em ambientes externos.

Dessa forma, o presente trabalho se distingue ao buscar não apenas
propor uma nova arquitetura de sensoriamento, mas também documentar as
dificuldades encontradas e as possíveis soluções para esses
obstáculos. 
